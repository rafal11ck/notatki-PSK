<<<<<<< HEAD
% Created 2022-12-21 Wed 16:48
=======
% Created 2023-01-18 Wed 09:46
>>>>>>> df49dd2 (China export)
% Intended LaTeX compiler: pdflatex
\documentclass[11pt]{article}
\usepackage[utf8]{inputenc}
\usepackage[T1]{fontenc}
\usepackage{graphicx}
\usepackage{longtable}
\usepackage{wrapfig}
\usepackage{rotating}
\usepackage[normalem]{ulem}
\usepackage{amsmath}
\usepackage{amssymb}
\usepackage{capt-of}
\usepackage{hyperref}
\author{Rafał Grot}
\date{\today}
\title{}
\hypersetup{
 pdfauthor={Rafał Grot},
 pdftitle={},
 pdfkeywords={},
 pdfsubject={},
 pdfcreator={Emacs 30.0.50 (Org mode 9.6)}, 
 pdflang={English}}
\begin{document}

\tableofcontents

\section{Pytatania jakie mogą być na egzaminie}
<<<<<<< HEAD
\label{sec:org62dca8d}
\subsection{Oapowiedzilność pracodawcy za nie przestrzeganie BHP?}
\label{sec:org13f6aa8}
=======
\label{sec:org31e3b30}
\subsection{Oapowiedzilność pracodawcy za nie przestrzeganie BHP?}
\label{sec:org7ff5fed}
>>>>>>> df49dd2 (China export)
\begin{itemize}
\item mandat od 500 do 5000zł
\item kara grzywny w wysokości od 1000zł do 30000zł
\item art 220 pozbawienie wolności do lat 3.
\end{itemize}
\subsection{bezpczeństwo techniczne}
<<<<<<< HEAD
\label{sec:org49cf96a}
=======
\label{sec:org953c4a8}
>>>>>>> df49dd2 (China export)
\begin{itemize}
\item obiekty, pomiesczenia pracy
\item maszyny, narzędzia, i urządzenia techniczne
\item energetyka, transport, procesy technologiczne
\item materiały i surowce
\item dokumentacja:
\begin{itemize}
\item budowlana
\item techniczna
\item ruchowa
\item instrukcjie
\end{itemize}
\end{itemize}
\subsection{Higiena pracy}
<<<<<<< HEAD
\label{sec:orgca11f3c}
=======
\label{sec:org1f47017}
>>>>>>> df49dd2 (China export)
\begin{itemize}
\item higiena pomieszczeń pracy
\item higiena pomiesczeń technologicznych
\item higiena osobista pracowników
\item zaplecze sanitarno higieniczne
\item środki higieny i zapezpieczenia higieniczne
\end{itemize}
\subsection{Prawna ochrona pracy}
<<<<<<< HEAD
\label{sec:org0bd9350}
=======
\label{sec:org27905ed}
>>>>>>> df49dd2 (China export)
\begin{itemize}
\item ochrona stosunku pracy
\item ochrona wynagrodzenia
\item ochrona warunków pracy
\item ochrona organizacji w pracy
\item ochrona pracy kobiet i młodocianych
\item ochrona świadczeń prcowniczych
\end{itemize}
\subsection{Czym się różni European conformance CE mark od China export}
<<<<<<< HEAD
\label{sec:org2fa742e}
=======
\label{sec:orgc31d2b5}
>>>>>>> df49dd2 (China export)
Trzeba narysować
\begin{figure}[htbp]
\centering
\includegraphics[width=.9\linewidth]{images/europeanConformanceCEChinaExport.jpg}
\caption{Po lewej EU, po prawej China export}
\end{figure}
\subsection{Organy nadzoru nad warunkami pracy}
<<<<<<< HEAD
\label{sec:org789f36e}
=======
\label{sec:orga0a5afa}
>>>>>>> df49dd2 (China export)
\begin{itemize}
\item Państowy
\begin{itemize}
\item sanepid
\item państwowa inspekcja pracy
\item urząd dozoru technicznego
\item państwowa straż pożarna
\item ochrona środowiska
\end{itemize}
\item Społeczny
\begin{itemize}
\item związki zawodowe
\item społeczna inspekcja pracy -- zakładowy społeczny inspektor pracy
\end{itemize}
\end{itemize}
\subsection{Trójkąt ochrony pracownika}
<<<<<<< HEAD
\label{sec:orga5920e5}
=======
\label{sec:org6b5b530}
>>>>>>> df49dd2 (China export)
\emph{od najmniej ważnych}
\begin{enumerate}
\item Środki ochrony indywidualnej
\item środki chorny organizacyjnej
\item środki ochrony zbiorowej
\item środki techniczne
\end{enumerate}
\subsection{Podstwowe parametry miejsca pracy}
<<<<<<< HEAD
\label{sec:orgb194246}
\subsubsection{wysokość}
\label{sec:orge1c56fd}
=======
\label{sec:org0d98a5f}
\subsubsection{wysokość}
\label{sec:org1403e45}
>>>>>>> df49dd2 (China export)
co najmniej 330cm,
\begin{itemize}
\item powieszchnia
\item dwa m\textsuperscript{2} wolnej powiesznich podłogi
\end{itemize}
\subsubsection{Oświetlenie}
<<<<<<< HEAD
\label{sec:org4656ca2}
\begin{enumerate}
\item Naturalne
\label{sec:orgde98694}
=======
\label{sec:org2b6f020}
\begin{enumerate}
\item Naturalne
\label{sec:org776f493}
>>>>>>> df49dd2 (China export)
powieszchnia okna:powiszchnia podłogi
\begin{itemize}
\item w pomiesczenieu przeznaczonym na pobyt ludzi 1:8.
\item w pomiesczeniach w, ktorych oświetlenie naturlane jest wymagane ze względu na przeznaczenie 1:12
\end{itemize}
\end{enumerate}
\subsubsection{Pomieszczenie bez okien wymagania:}
<<<<<<< HEAD
\label{sec:orga9ad77b}
=======
\label{sec:orgbe9f0a5}
>>>>>>> df49dd2 (China export)
\begin{itemize}
\item klimpatyzacja.
\item zgoda sanepidu.
\item zgoda inspekcji pracy.
\end{itemize}
\subsubsection{Gdize nie wolno stosować wentylacji mechanicznej}
<<<<<<< HEAD
\label{sec:org08271c0}
w pomieszczniach z paleńskiami na paliwo stałe, płynne lub urządzeniami gazowymi pobierające powietrze do spalania z pomieszczenia z grawitacyjnym odoprwadzeniem spalin przewodem stsowanie wentylacji mechanicznej jest zabronione.
\subsubsection{Kiedy pracodwaca zapenia pracownikom pomiesczenie do ogrzwania się}
\label{sec:org03f19a2}

\subsection{Kiedy pracodwaca zapenia pracownikom pomiesczenie do odpoczynku}
\label{sec:orgb5544e3}
=======
\label{sec:orgcb073b6}
w pomieszczniach z paleńskiami na paliwo stałe, płynne lub urządzeniami gazowymi pobierające powietrze do spalania z pomieszczenia z grawitacyjnym odoprwadzeniem spalin przewodem stsowanie wentylacji mechanicznej jest zabronione.
\subsubsection{Kiedy pracodwaca zapenia pracownikom pomiesczenie do ogrzwania się}
\label{sec:org34f5f6f}

\subsection{Kiedy pracodwaca zapenia pracownikom pomiesczenie do odpoczynku}
\label{sec:org6c7e529}
>>>>>>> df49dd2 (China export)
\begin{itemize}
\item jeżeli zatrudnia pracowników w pomiesczeniach ciasnych lub niskich
\item jeżeli zatrudnia pracowników w 30+ s c
\end{itemize}
\subsubsection{jakie warunki spełnia}
<<<<<<< HEAD
\label{sec:org40b035c}
=======
\label{sec:org0daa854}
>>>>>>> df49dd2 (China export)
\begin{itemize}
\item klimatyzownae
\item mieć miejsca siedzące
\end{itemize}
\subsection{Efekty z ergonomii}
<<<<<<< HEAD
\label{sec:org0b72d5b}
=======
\label{sec:org8ba4d18}
>>>>>>> df49dd2 (China export)
\begin{itemize}
\item zmniejszenie znaczenia różnic indywidualnych, tzn. im bardziej cechy maszyn, urządzeń i narzędia są przystosowane do możliwości człowieka.
\item zmniejszenie zmęczenia pracą
\item zwiększenie wydajności pracy
\item zapobieganie patologicznym skutkom wykonywania pracy, ograniczenie ilości chorób zawodowych.
\item zmniejszenie liczby wypadków przy pracy
\end{itemize}
\subsection{Czynniki materialnego środowiska pracy}
<<<<<<< HEAD
\label{sec:org5ec75dd}
=======
\label{sec:org4d31d24}
>>>>>>> df49dd2 (China export)
\begin{itemize}
\item Oświetlenie
\item Hałas, Drgania
\item[{Mikroklimat}] temperatura, wlgotność, ciśnienie, ruch powietrza, promieniowanie cieplne
\item[{Zaniczyszczenia powietrza}] pył albo związek chemiczny
\end{itemize}
\subsection{Czynniki antroptechniczne}
<<<<<<< HEAD
\label{sec:org13c3584}
=======
\label{sec:org63a829f}
>>>>>>> df49dd2 (China export)
\begin{itemize}
\item Postwa przy pracy.
\item Rytm i tempo pracy.
\item Przerwy w pracy.
\end{itemize}
\subsection{Na podstawie czego rozróżnami elementy sterownicze}
<<<<<<< HEAD
\label{sec:org28b94e2}
=======
\label{sec:org6875725}
>>>>>>> df49dd2 (China export)
\begin{itemize}
\item kształt
\item rozmiar
\item umejscowienie
\item barwa
\end{itemize}
\subsection{Co to jest centyl? / co student rozimie przez stosowanie zasad mair ograniczających}
<<<<<<< HEAD
\label{sec:org865de4f}
Centyl -- jest to punkt dla skali ocen, poniżej którego leży \(N\) \% wynikoów.
\subsection{Jakim wymiarom odpowiada 50 centylowa kobieta w stosunku do męszczyzn?}
\label{sec:org7fc958a}
\subsection{Co to jest dobrostan?}
\label{sec:org2226597}
Stan odczuwalny przez operatora podczas zgodnego z porzenaczeniem użytkowania maszyny, jeżeli dsykomofort, zmęcznie fizyczne i obicazenie psychidczne ograniczono do możliwego minimum dzięki uwzględnieniu zasad ergonomicznych
=======
\label{sec:org31ca647}
Centyl -- jest to punkt dla skali ocen, poniżej którego leży \(N\) \% wynikoów.
\subsection{Jakim wymiarom odpowiada 50 centylowa kobieta w stosunku do męszczyzn?}
\label{sec:org05c3595}
\subsection{Co to jest dobrostan?}
\label{sec:org5453cef}
Stan odczuwalny przez operatora podczas zgodnego z porzenaczeniem użytkowania maszyny, jeżeli dsykomofort, zmęcznie fizyczne i obicazenie psychidczne ograniczono do możliwego minimum dzięki uwzględnieniu zasad ergonomicznych
\subsection{Maksylamny wydatek energetyczny u kobiety, u mężczyzny.}
\label{sec:orgd33ee1f}
\subsection{Podaj przykład pracy z obciązeniem statycznym}
\label{sec:orgc6b39c8}
\subsection{Podstawowe parametry stanowiska pracy}
\label{sec:orgb2cc76b}
\begin{itemize}
\item 500 lux, biorowe
\item pani sprzątająca 100
\end{itemize}
>>>>>>> df49dd2 (China export)
\end{document}