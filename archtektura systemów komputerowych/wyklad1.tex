% Created 2022-10-04 Tue 18:41
% Intended LaTeX compiler: pdflatex
\documentclass[11pt]{article}
\usepackage[utf8]{inputenc}
\usepackage[T1]{fontenc}
\usepackage{graphicx}
\usepackage{longtable}
\usepackage{wrapfig}
\usepackage{rotating}
\usepackage[normalem]{ulem}
\usepackage{amsmath}
\usepackage{amssymb}
\usepackage{capt-of}
\usepackage{hyperref}
\author{Rafał Grot}
\date{\today}
\title{archtektura systemów komputerowych wykład 1 04.10.2022}
\hypersetup{
 pdfauthor={Rafał Grot},
 pdftitle={archtektura systemów komputerowych wykład 1 04.10.2022},
 pdfkeywords={},
 pdfsubject={},
 pdfcreator={Emacs 28.2 (Org mode 9.6)}, 
 pdflang={English}}
\begin{document}

\maketitle
\tableofcontents

\section{algorytmy}
\label{sec:org3a83d90}
\section{kodowania !}
\label{sec:orgac1e1fe}
\subsection{NKB}
\label{sec:orgee4d612}
\subsection{U1 i U2}
\label{sec:orge2d4b33}
\subsection{ZM}
\label{sec:org9889f04}
\subsection{Satło i zmiennopozycyjny}
\label{sec:orgbfe4e24}
\subsection{ASCII}
\label{sec:orgfb92935}
\subsection{UNICODE}
\label{sec:org27e019a}
Jest ich wiele
\subsection{Kodowanie Gray'a}
\label{sec:orge9445c8}
\section{Ręcznie kodowanie RSA}
\label{sec:orgdac72ab}
dla małych liczb
\section{komponenty}
\label{sec:org580c14a}
\subsection{moduł logiczny}
\label{sec:org61dd6f5}
układ logiczny, którey orperuje zgodnie z ale=gebrą Boola'a.Najprostrzymi modułami są bramki logiczne realizujące proste operacje: iloczynu, sumy ,różnicy symetrycznej i negacji.
\subsection{porjektownie}
\label{sec:orgfd8823e}
\begin{itemize}
\item metody klasyczne
\begin{itemize}
\item tablice Karnaugh
\end{itemize}
\item Języki HDL
(jeszcze żyją)
\end{itemize}
\subsection{funktory logiczne}
\label{sec:orgc90b691}
Są realizacją sprzętową matematycznego modelu funkcji logicznej.
dzielimy od złożności dzilimy na:
\begin{description}
\item[{Układy małe skali integracji (SSI)}] relizują podstawowe operacjie logiczne: AND, OR, NOT, XOR
\item[{średniej skali integracji (MSI)}] 
\end{description}
\section{przesyłanie informacji}
\label{sec:orgc8cd0a1}
\subsection{komunikacja}
\label{sec:org7ac4f78}
\subsubsection{W komputerach klasy PC}
\label{sec:org2d7f502}
Odbywa się na dwóch platformach
\begin{enumerate}
\item sprzętowej
\label{sec:org97411e0}
\begin{description}
\item[{wymiana danych między modułami komputera}] np CPU i pamięci
\item transmisje danych z/do pamięci masowej
\item przesyłanie z/do urządzeń peryferyjnyc
\end{description}
\item softwareowej
\label{sec:org7a470b0}
\begin{itemize}
\item transmitowanie pamięci między aplikacjami
\end{itemize}
\begin{enumerate}
\item przesyłanie danych między systemami komputerami (sieci komputerowe)
\label{sec:org7dc5912}
\end{enumerate}
\item architektury połączeń
\label{sec:org29dcb58}
\begin{enumerate}
\item magistrali
\label{sec:orgefe5a31}
\item gwiazdy
\label{sec:orgd9043b3}
\item pierścienia
\label{sec:org26e2da4}
\item siatki
\label{sec:org4df6d58}
\end{enumerate}
\end{enumerate}
\section{Komputer klasy PC}
\label{sec:orgc141537}
\subsection{Posiada cechy:}
\label{sec:orgbcd6280}
\begin{description}
\item[{porgramowalność}] umożliwa uruchomienie dowolnego oprogramowania lub jego stworzenia
\item[{uniwersalne zastosowania}] \begin{description}
\item[{typowo użytkowe}] jako maszyna do pisania i gormadzenia danych, jako narzędzie obliczeniowe, wpracy biurowej
\item o charkaterze rozrywkowym
\item[{badawczo-naukowe}] w naukach ścisłych, w medycynie, w farmacji
\item[{wspomagające projektowanie}] np narzędzia CAD, CAE, CAI, CAM, CASE itp.
\end{description}
\item modularność konstrukcji
\end{description}
\subsection{komponenty}
\label{sec:org38dd8cc}
\subsubsection{monitor}
\label{sec:orgd17578c}
\subsubsection{klawaitura}
\label{sec:orgb3251d5}
\subsubsection{urządzenie wskazujące (mysz)}
\label{sec:org02a3b24}
\subsection{jednostka centralną}
\label{sec:orgc8a3677}
\subsubsection{CPU}
\label{sec:orgfcc519b}
\subsubsection{płytę główną}
\label{sec:org36ae1ff}
\subsubsection{kart roszerzające}
\label{sec:org209b015}
NP:
\begin{itemize}
\item karty graficzne
\item karty muzyczne
\item karty sieciowe
\end{itemize}
\subsubsection{pamięć operacyjna}
\label{sec:orgeb7fb7e}
wykonana z modółów pamięci RAM
\end{document}
