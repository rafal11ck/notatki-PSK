% Created 2023-01-20 Fri 11:07
% Intended LaTeX compiler: pdflatex
\documentclass[11pt]{article}
\usepackage[utf8]{inputenc}
\usepackage[T1]{fontenc}
\usepackage{graphicx}
\usepackage{longtable}
\usepackage{wrapfig}
\usepackage{rotating}
\usepackage[normalem]{ulem}
\usepackage{amsmath}
\usepackage{amssymb}
\usepackage{capt-of}
\usepackage{hyperref}
\author{Rafał Grot}
\date{1.20.2023}
\title{Wyklad 11 Generatory liczb pseudolosowych}
\hypersetup{
 pdfauthor={Rafał Grot},
 pdftitle={Wyklad 11 Generatory liczb pseudolosowych},
 pdfkeywords={},
 pdfsubject={},
 pdfcreator={Emacs 30.0.50 (Org mode 9.6)}, 
 pdflang={English}}
\begin{document}

\maketitle
\tableofcontents

$$R = \{\underbrace{a_0, a_{1}, \ldots , a_{N-1}}_{N \text{ liczb} }\}, a \in N$$
\section{Generator liniowy kongruencyjny (G.L.K)}
\label{sec:org14f8ad7}
$$x_{i+1} = (a \cdot x_{i} ;+b ) \text{ mod } N, x_0 = C \text{ -- seed}$$
2\textsuperscript{31} \(\cdot\) 1, 2\textsuperscript{32}

\noindent\rule{\textwidth}{0.5pt}
$$x_{i+M} = ( \underbrace{ a_{M} x_{i+M-1} + a_{M-2}x_{i+M-2} \ldots + a_{0}x_{i} }_{M \text{ składników }}) \text{ mod } N$$

\section{Okres generatora}
\label{sec:orga356c1b}
\(L = N^{M} -1\) -- ogólnie dla G.L.K
\(L = N -1\)
\subsection{Np.}
\label{sec:orgea7a155}
$$x_{i+1} = (3 x_; +1) \mod 7$$
$$\underbrace{x_0 = 0,x_1=1, x_2 =4 , x_3=6, x_4 = 5, x_5 = 2}_{L = 6}, x_6 = 0$$
\section{Prawdopodobieństwo wystąpienia liczby \(x_{I+1}\) w sekwencji R to p:}
\label{sec:org7c025b1}
\(p_i = \frac{1}{L}\) -- przypadek idelny
\subsection{generator o rozkladzie gęstości pradopodobieństwa \(p_i\) równomiernym}
\label{sec:org33dc136}
$$U[0,L] \to \sum_{i=0}^{L}p_{i} =1, p_{i} = p_{j}, i \neq j, i,j \in {[0,L]} $$
\subsection{Test wartosci średniej}
\label{sec:org54f1d57}
\subsubsection{Dla gen. o rozkładzie \(U[0,1]\)}
\label{sec:orgb2678b7}
$$\frac{1}{L} \sum_{i=0}^{L} x_{i} = 0.5$$
\subsection{Test wriacji}
\label{sec:orgb4f4f41}
$$\frac{1}{L} \sum_{i=0}^{L} (x_{i} -\bar{x})^{2}; = 0$$
\subsection{Test PI}
\label{sec:org961223a}
$$P_i[x_i, x_{i+1}]$$
\(L_{\square} =\) liczba \(P_i\), wnętrza\(_{\square}\)
\(L_{\circ} =\) liczba \(P_i\), wnętrza\(_{\circ}\)

$$\frac{L_{\square}}{L_{\circ}} = \frac{P_{\square}}{P_{\circ}}$$
$$P_{\square} = a \cdot a$$

$$P_{\circ} = \pi(\frac{a}^{2} })^2
= \pi \frac{a^2}{4}$$

$$\frac{P_{\square}}{P_{\circ}}
= \frac{a^2}{\pi \frac{a^2}{4}}
= \frac{4}{\pi} \to \frac{L_{\square}}{L_{\circ}}
= \frac{4}{\pi}$$
\end{document}