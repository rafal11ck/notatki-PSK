% Created 2022-10-08 Sat 15:48
% Intended LaTeX compiler: pdflatex
\documentclass[11pt]{article}
\usepackage[utf8]{inputenc}
\usepackage[T1]{fontenc}
\usepackage{graphicx}
\usepackage{longtable}
\usepackage{wrapfig}
\usepackage{rotating}
\usepackage[normalem]{ulem}
\usepackage{amsmath}
\usepackage{amssymb}
\usepackage{capt-of}
\usepackage{hyperref}
\author{Rafał Grot}
\date{\today}
\title{Wprowadzenie do systemów liczbowych}
\hypersetup{
 pdfauthor={Rafał Grot},
 pdftitle={Wprowadzenie do systemów liczbowych},
 pdfkeywords={},
 pdfsubject={},
 pdfcreator={Emacs 28.2 (Org mode 9.6)}, 
 pdflang={Polish}}
\begin{document}

\maketitle
\tableofcontents

\large

\textbf{MACIE OPANOWAC NKB I U2}
wstęp do infrmatyki moodle bez hasła

\section{System liczbowy o podstawie \(R\)}
\label{sec:orgd220aac}
\(R \in \mathbb{N}\) \\
Alfabet: \(A = \{ \hat{a}_0, \hat{a}_1, \dots, \hat{a}_0 \}\) \\
\(|A| = R\)
\subsection{DEC \(\to\) SR}
\label{sec:org110a1ff}
\begin{latex}
\begin{tabular}{c|c|c}
    $X_{\text{DEC}}:R =$& Wynik & Reszta \\
    \hline
    $X_{0}:R$ & W_0 & R_{0} \\
    $X_{1}:R$ & W_1 & R_{1} \\
    $X_{2}:R$ & W_2 & R_{2} \\
    \vdots & \vdots & \vdots \\
    $X_{N-2}:R$ & W_{N-1} & R_{N-1} \\
    $X_{N-1}:R$ & 0 & R_{N} \\
\end{tabular}
\(\uparrow\)
\end{latex}
Odczytujemy od dołu.

$$X_{\text{DEC}} = Y_R = ( R_N R_{N-1} \dots R_{1} R_{0})_{R}$$
\subsubsection{przykład}
\label{sec:org1521b6d}
\(110_{\text{DEC}} \xrightarrow{?} Y_{2}\)

\begin{latex}

\begin{tabular}{c c|c}
110_{\text{DEC}}:2= & 55 & 0 \\
55 : 2 = & 27 & 1 \\
27 : 2 = & 13 & 1 \\
13 : 2 = & 6 & 1 \\
6 : 2 = & 3 & 0 \\
3 : 2 = & 1 & 1 \\
1 : 2 = & 0 & 1 \\
\end{tabular}
\(\uparrow\)

$$Y_2=1101110_2=110_{10}$$
\end{latex}
\subsection{SR \(\to\) DEC}
\label{sec:org2681616}
$$X_{\text{DEC}} = \sum_{i=0}^{N} a_i \cdot R^{i}, a_{i}: \in A$$

\subsubsection{Przykład}
\label{sec:org9a4e022}
\(1101110_{2} \xrightarrow{?}\) DEC
$$X_{\text{DEC}} = 0 \cdot 2^0 + 1 \cdot 2^1 + 1 \cdot 2^2 + 1 \cdot 2^3 + 0 \cdot 2^4 + 1 \cdot 2^5 + 1 \cdot 2^6 = 0 + 2 + 4 + 8 + 0 + 32 + 64 = 110_{10} $$

\section{System ZN (Znak moduł)}
\label{sec:org0165023}
$$\text{L}_{\text{ZN}} = (b_n \underbrace{ a_{N-1} a_{N-2} \dots  a_1 a_0}_{L_{\text{NKB}} \geq 0} )_{\text{ZN}}$$
$$a_i \in A , i=0,1,\dots,N-1$$
$$b_N \in \{0,1\}$$
gdzie
\begin{itemize}
\item \(b_N=0\) oznacza \(L\geq0\)
\item \(b_N=1\) oznacza \(L<0\)
\end{itemize}
\section{NKB}
\label{sec:orge886311}
$$\text{L}_{\text{NKB}}=(a_{N-1}a_{N-2} \dots a_1 a_0)_{\text{NKB}}$$
\subsection{NKB \(\to\) DEC}
\label{sec:org569deb9}
$$\text{L}_{\text{DEC}} = \sum_{i=0}^{N-1} a_i \cdot 2^i $$
\subsubsection{Przykład}
\label{sec:org71af190}
$$\underset{64}{1}10\underset{8}{1}1\underset{2}{1}0_{\text{NKB}} = 110_{\text{DEC}}$$
\subsection{DEC \(\to\) NKB}
\label{sec:org4435383}
Użyj algorytmu \ref{sec:org110a1ff} dla \(R = 2\)
\section{System U2 (dopełneniowy do 2)}
\label{sec:orgbb74f9d}
$$\text{L}_{\text{U2}}=(a_{N-1}a_{N-2} \dots a_1 a_0)_{\text{U2}}$$

\(a_{N-1}}\) -- waga ujemna
\subsection{\(\test{L}_{\text{U2}} \to\) DEC}
\label{sec:org5ed86ed}
Coś mi się wydaje że to powinno być \(L_{\text{DEC}}\) ale tak jest(było) na tablicy.

$$\text{L}_{\text{U2}}= {-a_{N-1}} \cdot 2^{N-1} + \sum_{i=0}^{N-1} a_i \cdot 2^i$$
\subsubsection{przykład}
\label{sec:orgb141ff2}
$$\underset{-64}{1}10\underset{8}{1}1\underset{2}{1}0_{\text{U2}} = -18_{\text{DEC}}$$
\subsection{DEC \(\to\) U2}
\label{sec:org07f4226}
\begin{enumerate}
\item Użyj algorytmu \ref{sec:org110a1ff} dla \(R=2\) \emph{(Czyli tak samo jak \ref{sec:org4435383})}

\(\text{L} = (a_{N-1} a_{N-2} \dots a_1 a_0)\)
\item Dostaw "0" do najbardziej znaczącej cyfry
\end{enumerate}

\(\text{L} = (0 a_{N-1} a_{N-2} \dots a_1 a_0)\)
\begin{enumerate}
\item dla \(\text{L} \geq 0 \quad\) KONIEC
\item dla \(\text{L} < 0 \quad\)
\end{enumerate}
\begin{latex}
\begin{tabular}{llllllll}
$\text{L} &= (&0 &a_{N-1} &a_{N-2} &\dots &a_1 &a_0)$ \\
&+ &0 &0 &0 &\dots &0 &1\\
\hline
$\text{L}_{\text{U2}} &= &(b_N &b_{N-1} &b_{N-1} &\dots &b_{2} &b_{1}$
\end{tabular}
\end{latex}
\end{document}
