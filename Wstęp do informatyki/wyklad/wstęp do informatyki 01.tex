% Created 2022-10-07 Fri 23:41
% Intended LaTeX compiler: pdflatex
\documentclass[11pt]{article}
\usepackage[utf8]{inputenc}
\usepackage[T1]{fontenc}
\usepackage{graphicx}
\usepackage{longtable}
\usepackage{wrapfig}
\usepackage{rotating}
\usepackage[normalem]{ulem}
\usepackage{amsmath}
\usepackage{amssymb}
\usepackage{capt-of}
\usepackage{hyperref}
\author{Rafał Grot}
\date{\today}
\title{Wprowadzenie do systemóœ liczbowych}
\hypersetup{
 pdfauthor={Rafał Grot},
 pdftitle={Wprowadzenie do systemóœ liczbowych},
 pdfkeywords={},
 pdfsubject={},
 pdfcreator={Emacs 28.2 (Org mode 9.6)}, 
 pdflang={English}}
\begin{document}

\maketitle
\tableofcontents

\large

\textbf{MACIE OPANOWAC NKB I U2}
wstęp do infrmatyki moodle bez hasłą

\section{System liczbowy o podstawie R \(R \in N\)}
\label{sec:org2d55117}
Alfabet: \(A = \{ \hat{a}_0, \hat{a}_1, \dots, \hat{a}_0 \} , |A| = R\\\)
\subsection{DEC -> SR}
\label{sec:org717f9ce}
\begin{latex}
\begin{tabular}{c|c|c}
    $X_{DEC}:R =$& Wynik & Reszta \\
    \hline
    $X_{0}:R$ & W_0 & R_{0} \\
    $X_{1}:R$ & W_1 & R_{1} \\
    $X_{2}:R$ & W_2 & R_{2} \\
    \vdots & \vdots & \vdots \\
    $X_{N-2}:R$ & W_{N-1} & R_{N-1} \\
    $X_{N-1}:R$ & 0 & R_{N} \\
\end{tabular}
\(\uparrow\)
\end{latex}
Odczytujemy od dołu.

$$X_{DEC} = Y_R = ( R_N, R_{N-1}, \dots, R_{1}, R_{0})_{R}$$
\subsubsection{\(110_{DEC} \xrightarrow{?} Y_{2}\)}
\label{sec:org3e57a83}
\begin{latex}

\begin{tabular}{c c|c}
110_{DEC}:2= & 55 & 0 \\
55 : 2 = & 27 & 1 \\
27 : 2 = & 13 & 1 \\
13 : 2 = & 6 & 1 \\
6 : 2 = & 3 & 0 \\
3 : 2 = & 1 & 1 \\
1 : 2 = & 0 & 1 \\
\end{tabular}
\(\uparrow\)

$$Y_2=1101110_2=110_{10}$$
\end{latex}
\subsection{SR \(\to\) DEC}
\label{sec:orgae9db52}
$$X_{DEC} = \sum_{i=0}^{N} a_i \cdot R^{i}, a_{i}: \in A$$

np. \(1101110_{2} \xrightarrow{?} DEC\)
$$X_{DEC} = 0 \cdot 2^0 + 1 \cdot 2^1 + 0 \cdot 2^2 + 1 \cdot 2^3 + 0 \cdot 2^4 + 1 \cdot 2^5 + 1 \cdot 2^6 = 0 + 2 + 4 + 8 + 0 + 32 + 64 = 110_{10} $$
\end{document}
