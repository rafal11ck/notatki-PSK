% Created 2022-10-07 Fri 17:59
% Intended LaTeX compiler: pdflatex
\documentclass[11pt]{article}
\usepackage[utf8]{inputenc}
\usepackage[T1]{fontenc}
\usepackage{graphicx}
\usepackage{longtable}
\usepackage{wrapfig}
\usepackage{rotating}
\usepackage[normalem]{ulem}
\usepackage{amsmath}
\usepackage{amssymb}
\usepackage{capt-of}
\usepackage{hyperref}
\author{Rafał Grot}
\date{\today}
\title{Wprowadzenie do systemóœ liczbowych}
\hypersetup{
 pdfauthor={Rafał Grot},
 pdftitle={Wprowadzenie do systemóœ liczbowych},
 pdfkeywords={},
 pdfsubject={},
 pdfcreator={Emacs 28.2 (Org mode 9.6)}, 
 pdflang={English}}
\begin{document}

\maketitle
\tableofcontents

\textbf{MACIE OPANOWAC NKB I U2}
wstęp do infrmatyki moodle bez hasłą
\section{System liczbowy}
\label{sec:org03771cf}
\subsection{System liczbowy o podstawie R: \(\R \in N\)}
\label{sec:orgb421e36}
Alfabet: \(\A = {\hat{a}_1,\hat{a}_2,...,\hat{a}_R}, |A|=R\)
\(SR \to DEC\):
\(x_{DEC} : R = W | Reszta\)
\$X\textsubscript{DEC} : R | \(W_0\) | \(R_{0}\)
\(X_{0} :R\) | \(W_1\) | \(R_{1}\)
\(X_{1} :R\) | \(W_2\) | \(R_{2}\)
\(W_{W-2}\) :R | \(W_{W-1}\) | \(R_{N-1}\)
\(W_{W-1}\) :R | \(0\) | \(R_{N}\)
$$X_{DEC}=Y_R=(R_R, R_{R-1},...,R_1,R_0)$$

\(110_{DEC} : 2 = 55| 0\)
\(55_{DEC} : 2  = 27| 1\)
\(27_{DEC} : 2  = 13| 1\)
\(13_{DEC} : 2  = 6| 1\)
\(6_{DEC} : 2  = 3| 0\)
\(3_{DEC} : 2  = 1| 1\)
\(1_{DEC} : 2  = 1| 1\)

\subsection{\(SR \to DEC\)}
\label{sec:org56c9752}
\(X_{DEC} = \foreach{N}{i=0}a_i*R , a_i \in A\)
np:
\$lX\textsubscript{DEC} = 0 * 2\textsuperscript{0} + 1 * 2\textsuperscript{1} + 1*2\textsuperscript{2} + 0 * 2\textsuperscript{3} + 0 * 2\textsuperscript{4} + 1*2\^{}\% + 1*2\textsuperscript{6} = 0 + 2 7+ 4 + 8 + 0 + 32 + 64 = 110\$\$
\section{Reperezentacja liczb ujemnych}
\label{sec:org5bf9468}
\subsection{System ZM (Znak moduł)}
\label{sec:orgb7ea601}
\(L_{ZM} = (b_n a_{aN-1}, a_{N-2}, ... , a_{1}, a_0)_{ZM}), a\such \in A, i-a_0,...a_{N-1}\)
\(b_N \in {0,1}\) gdzie \(b_n=0\) oznacza \(L > 0\)
\(b_N=1\) oznacza \(L < 0\)
\subsection{System U2 (dopełniniowy do 2)}
\label{sec:org9c7bf30}
\subsubsection{BIN \(\to\) DEC}
\label{sec:org8bd6c27}
\begin{enumerate}
\item U2
\label{sec:org6a385a1}
\(L_{U2}= (a_{N-1+} a_{N-2} ... a_1 a_0)_{U2}\)
\(U2 \to DEC\)
\(L_{U2}= a_{N-1}*2^{N-1} + \sum{u-2}{i=0}a_i*2\)

\(L_{DEC}\)
\item MKB
\label{sec:org0c06a06}
\(L_{MKB}= (a_{N-1+} a_{N-2} ... a_1 a_0)_{NKB}\)

\(1101110_{U2} = -18\)
\(1101110_{NKB} = 110_{DEC}\)
\end{enumerate}
\subsubsection{DEC \(\to\) BIN}
\label{sec:org35731af}
\begin{enumerate}
\item DEC \(\to\) NKB
\label{sec:orgb43789f}
Użyj algorytmu DEC \(\to\) SR dla \(R=2\)
\begin{enumerate}
\item Użyj algorytmu DEC \(\to\) SR dla \(R=2\)
\end{enumerate}
\(L=(a_{N-1} a_{N-2} ... a_1 a_0)\)
\begin{enumerate}
\item dodaj ``0'' do najbardziej znaczącej cyfry
\end{enumerate}
\(L_{NKB}=(a_{N-1} a_{N-2} ... a_1 a_0)_{NKB}\)
\(L_{NKB}=(a_{N-1} a_{N-2} ... a_1 a_0)_{NKB}\)
\(L_{NKB}= 0 a_{N-1} a_{N-2} ... a_1 a_0\)
\begin{enumerate}
\item dla \(L_{DEC} >= 0\) KONIEC; \$dla \$L\textsubscript{DEC} < 0  ?????
\end{enumerate}
\(L_{U2}=(B_w b_{w-1} b_{w-2} ... b_1 b_0)_{U2}\)

\(1101110_{NKB}\)
\(110_{DEC} = 01101110_{U2}\)
\(-110_{DEC}=01101110\)
100100001
+00000001

\noindent\rule{\textwidth}{0.5pt}
\(10010010_{U2} = -110_{DEC}\)

DEC \(\to\) NKB
NKB \(\to\) U2
zmiana znaku \textbf{L<0}
\end{enumerate}
\end{document}
