% Created 2022-10-25 Tue 10:48
% Intended LaTeX compiler: pdflatex
\documentclass[11pt]{article}
\usepackage[utf8]{inputenc}
\usepackage[T1]{fontenc}
\usepackage{graphicx}
\usepackage{longtable}
\usepackage{wrapfig}
\usepackage{rotating}
\usepackage[normalem]{ulem}
\usepackage{amsmath}
\usepackage{amssymb}
\usepackage{capt-of}
\usepackage{hyperref}
\author{Rafał Grot}
\date{\today}
\title{Wstęp Do Informatyki 03. Podstawowe typy danych}
\hypersetup{
 pdfauthor={Rafał Grot},
 pdftitle={Wstęp Do Informatyki 03. Podstawowe typy danych},
 pdfkeywords={},
 pdfsubject={},
 pdfcreator={Emacs 28.2 (Org mode 9.6)}, 
 pdflang={English}}
\begin{document}

\maketitle
\tableofcontents

\begin{itemize}
\item Liczby całkowite
\item liczby rzeczywiste
\end{itemize}
\section{Operacje na liczbach całkowych}
\label{sec:org50ee3f3}
\subsection{Arytmetyczne}
\label{sec:org9b47dee}
\begin{center}
\begin{tabular}{|c|c|c|c|c|}
\hline
+ & - & / & * & \%\\
\hline
ADD & SUB & DIV & POW & MOD\\
\hline
\end{tabular}
\end{center}
\subsubsection{+}
\label{sec:org66533d8}
\begin{center}
\begin{tabular}{lrrrrrrrll}
 & 1 & 1 & 1 & 1 & 1 & 1 & 1 &  & \\
 &  &  &  &  &  &  &  &  & \\
 & 1 & 1 & 0 & 1 & 1 & 0 & 0 & 1\textsubscript{NKB(8)} & =217\textsubscript{DEC}\\
+ & 0 & 0 & 1 & 0 & 1 & 1 & 1 & 1\textsubscript{NKB(8)} & =47\textsubscript{DEC}\\
\hline
 & 0 & 0 & 0 & 0 & 1 & 0 & 0 & 0\textsubscript{NKB(8)} & =264\textsubscript{DEC} - 256\textsubscript{DEC} = 8\textsubscript{DEC}\\
\end{tabular}
\end{center}
\subsubsection{-}
\label{sec:org174c0ea}
\begin{center}
\begin{tabular}{lrrrrrrrll}
 &  &  &  &  & 1 &  &  &  & \\
 &  & 0 & 2 & 0 & 2 & 2 &  &  & \\
 &  &  &  &  &  &  &  &  & \\
 & 1 & 1 & 0 & 1 & 1 & 0 & 0 & 1\textsubscript{NKB(8)} & =217\textsubscript{DEC}\\
- & 0 & 0 & 1 & 0 & 1 & 1 & 1 & 1\textsubscript{NKB(8)} & =47\textsubscript{DEC}\\
\hline
 & 1 & 0 & 1 & 0 & 1 & 0 & 1 & 0\textsubscript{NKB(8)} & =170\textsubscript{DEC}\\
\end{tabular}
\end{center}
\subsection{Logiczne}
\label{sec:orga5700c1}
\begin{center}
\begin{tabular}{|c|c|c|}
\hline
\&\& & \textbar\textbar & !\\
\hline
AND & OR & NOT\\
\hline
Koniunkcja & Alternatywa & \\
\hline
\end{tabular}
\end{center}
\subsubsection{AND}
\label{sec:org20a7a06}

\begin{enumerate}
\item A=1, B=2
\label{sec:org6b7aa82}

\(A\) \&\& \(B =\) TRUE

\textbf{Bo}

\(A=1 \to A =\) TRUE

\(B=2 \to B =\) TRUE

TRUE \&\& TRUE \(=\) TRUE
\begin{verbatim}
int a = 1;
int b = 2;
if (a && b) {}
\end{verbatim}
\item \(A=0 , B=0\)
\label{sec:orge4d7a99}

\(A\) \&\& \(B =\) FALSE \\
FALSE \&\& FALSE \(=\) FALSE
\end{enumerate}
\subsubsection{OR}
\label{sec:orgb68d1bd}
\(A=1, B=2\) \\
\(A\) || \(B\) = TRUE
\subsubsection{NOT}
\label{sec:orgf331da8}

\(A=0\) \\
\(!A=\) TRUE\\
\(A=\) FALSE

\begin{verbatim}
int a = false;
int b = true;
\end{verbatim}
a=0 \\
b \(\neq\) 0
\subsection{Bitowe}
\label{sec:org1234aca}
\begin{center}
\begin{tabular}{|c|c|c|c|c|c|}
\hline
\& & \textbar & \textasciitilde{} & \(\ll\) & \(\gg\) & \^{}\\
\hline
AND & OR & NOT & SHL & SHR & XOR\\
\hline
\end{tabular}
\end{center}
\subsubsection{AND}
\label{sec:org75e49e7}

\(A=1\) \\
\(B=2\) \\
\(A\) \& \(B=0\)

\begin{center}
\begin{tabular}{rcccc}
\(A=\) & 0 & 0 & 0 & 1\\
\(B=\) & 0 & 0 & 1 & 0\\
\hline
\(A\) \& \(B=\) & 0 & 0 & 0 & 0\\
\end{tabular}
\end{center}

Częsty błąd:
\begin{verbatim}
int a=1;
int b=2;
if (a && b) {} // dobrze
if (a & b) {}  // źle
\end{verbatim}

\subsubsection{\(\ll\)}
\label{sec:orga73ffbf}

\(A=1\) \\
\(B=2\) \\
\(A \ll B = 4\)

$$A= a_{N-1} a_{N-2} \dots a_1 a_0 $$
$$A \ll B = \underbrace{a_{N-1} a_{N-2} \dots a_1 a_0}_{N-B \textit{bitów}} \underbrace{0 \dots 0}_{B \textit{bitów}}$$

\subsubsection{\(\gg\)}
\label{sec:org654cbf7}

$$A \gg B = ?$$

$$A= a_{N-1} a_{N-2} \dots a_1 a_0 _{ \underset{ \text{U2}}{\text{NKB}} } $$
\begin{enumerate}
\item NKB
\end{enumerate}
$$A_{\text{NKB}} \gg B = \underbrace{0 \dots 0}_{B \textit{bitów}} \underbrace{a_{N-1} a_{N-2} \dots a_{N+1} a_{N}}_{ {N-B} \textit{bitów}} _\text{NKB} $$

\begin{enumerate}
\item U2
\end{enumerate}
$$A_{\text{U2}} \gg B = \underbrace{a_{N-1} \dots a_{N-1}}_{B \textit{bitów}} \underbrace{a_{N-1} a_{N-2} \dots a_{N+1} a_{N}}_{ {N-B} \textit{bitów}} _\text{U2} $$

\hline

\begin{verbatim}
char a = -2;
char b = 2;
char c = a >> b;
\end{verbatim}
\begin{center}
\begin{tabular}{lrrrrrrlrl}
\(a =\) & 1 & 1 & 1 & 1 & 1 & 1 & 0\textsubscript{U2} & 1 & \\
\(c =\) & 1 & 1 & 1 & 1 & 1 & 1 & 1\textsubscript{U2} & 1 & \(= -1_{\text{DEC}}\)\\
\end{tabular}
\end{center}

\hline
\begin{verbatim}
unsigned char a = -2;     // a=254(DEC)
unsigned char b = 2;
unsigned char c = a >> b; // 63(DEC)
char d = c;               // 63(DEC)
\end{verbatim}

\begin{center}
\begin{tabular}{llrrrrrrrrl}
 & a = & 1 & 1 & 1 & 1 & 1 & 1 & 1 & 0\textsubscript{NKB} & \\
 & c = & 0 & 0 & 1 & 1 & 1 & 1 & 1 & 1 & \(= 63\)\\
\end{tabular}
\end{center}

\subsubsection{Odczyt bitów (badanie bitów)}
\label{sec:org5bc6811}
\begin{verbatim}
char A = 10;
\end{verbatim}
\(A=00001010_{U2}\) \\
\(k=3\) bit \\
\(B=00000100_{U2}\)

\(A\) \& \(B\) = 00000000\textsubscript{\text{U2}} = 0\textsubscript{\text{DEC}} \(\to\) FAŁSZ

\(A\) \& \(4\) = 00000000\textsubscript{\text{U2}} = 0\textsubscript{\text{DEC}} \(\to\) FAŁSZ
\subsubsection{Ustawianie bitów}
\label{sec:orgeb26dd5}

\(k=3\)

\begin{center}
\begin{tabular}{rll}
\(A=\) & \(00001010_{\text{U2}}\) & \(= 10_{\text{DEC}}\)\\
\(B=\) & \(00000100_{\text{U2}}\) & \(= -3_{\text{DEC}}\)\\
\hline
\(A \textbar B =\) & \(00001110_{\text{U2}}\) & \(= 14_{\text{DEC}}\)\\
\end{tabular}
\end{center}

\subsubsection{Zerowanie bitów}
\label{sec:org968e748}

\(k=3\)

\begin{center}
\begin{tabular}{rll}
\(A=\) & \(00001010_{\text{U2}}\) & \(= 10_{\text{DEC}}\)\\
\(B=\) & \(11111101_{\text{U2}}\) & \(= -3_{\text{DEC}}\)\\
\hline
\(A \& B =\) & \(00001000_{\text{U2}}\) & \(= 8_{\text{DEC}}\)\\
\end{tabular}
\end{center}

\subsubsection{Negacja bitów}
\label{sec:org9d7044c}

\(k=3\)

\begin{center}
\begin{tabular}{rll}
\(A=\) & \(00001010_{\text{U2}}\) & \(= 10_{\text{DEC}}\)\\
\(B=\) & \(00000100_{\text{U2}}\) & \(= 4_{\text{DEC}}\)\\
\hline
\(A \textasciicircum B =\) & \(00001110_{\text{U2}}\) & \\
\end{tabular}
\end{center}


\hline

\begin{lang}
int A = 256;
\end{lang}

\begin{center}
\begin{tabular}{l|l|l|l|l|l|}
 &  & B & G & R & \\
\hline
bity \(\to\) &  &  &  &  & \\
\hline
 & 32 & 24 & 16 & 8 & 0\\
\end{tabular}
\end{center}

\begin{verbatim}
int R = A & 255;         // A & 0xFF
int G = (A >> 8) & 255;
int B = (A >> 16) & 255;
\end{verbatim}

\hline

\begin{verbatim}
int R = 5;
int G = 10;
int B = 20;
int A = (A<<16) | (G<<8) | R;
\end{verbatim}

\hline

\begin{verbatim}
int a = 2;
int b = a * 2;
int c = a << 1 ; // 2^k
\end{verbatim}

\hline

\begin{verbatim}
if(a % 2 != 0) {} // Z automatu nizdany egzamin w tym semestrze
if(a & 1) {}

if(a > 0);
\end{verbatim}
\end{document}
