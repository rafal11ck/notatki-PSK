% Created 2023-01-13 Fri 14:28
% Intended LaTeX compiler: pdflatex
\documentclass[11pt]{article}
\usepackage[utf8]{inputenc}
\usepackage[T1]{fontenc}
\usepackage{graphicx}
\usepackage{longtable}
\usepackage{wrapfig}
\usepackage{rotating}
\usepackage[normalem]{ulem}
\usepackage{amsmath}
\usepackage{amssymb}
\usepackage{capt-of}
\usepackage{hyperref}
\author{Rafał Grot}
\date{\today}
\title{Wyklad10, złożoność obliczeniowa algorytmów}
\hypersetup{
 pdfauthor={Rafał Grot},
 pdftitle={Wyklad10, złożoność obliczeniowa algorytmów},
 pdfkeywords={},
 pdfsubject={},
 pdfcreator={Emacs 30.0.50 (Org mode 9.6)}, 
 pdflang={English}}
\begin{document}

\maketitle
\tableofcontents


\section{DLL TSP}
\label{sec:orgf0ffcaa}
\texttt{Nazwisko\_Imię\_GRXXX.dll}
\begin{description}
\item[{Tabelka eksportu}] \texttt{FindRoad}.
\end{description}

A nie:
\begin{itemize}
\item \texttt{\_FindRoad}
\item \texttt{\_FindRoad@12}
\item \texttt{FindRoad@12}
\end{itemize}

Przed deklracją funkcji exportowej
\texttt{extern "C" void \_\_stdcall FindRoad}\ldots{}
\begin{itemize}
\item w zależności od kompilatora \texttt{\_\_stdcall} trzeba wywalić.
\end{itemize}

\texttt{tdumb -ee nazwa.dll}
\section{Złożoność obliczeniowa algorytmów}
\label{sec:org8d7a141}
\subsection{Złożoność pamięciowa}
\label{sec:orgb8a5d8a}

\subsection{Złożoność czasowa}
\label{sec:org89c64fa}
Jak szybko rośnie zapotrzebowanie algorytmu wraz ze wzrostem rozmiaru zadania.\\\empty
\(N\) -- rozmiar zadania algortymicznego.\\\empty
\(f(N)\) -- funkcja złożoności obliczeniowej.
\subsection{Typowe funkcjie złożoności obliczeniowej}
\label{sec:org6556dbb}
\subsubsection{Funkcja Stała}
\label{sec:orgbc81672}
\(F(N) = A, A  =const\) \\\empty
\(O(1)\)
\begin{enumerate}
\item \(O(1)\)
\label{sec:org6d67e2c}
$$O(1) = O(1) + O(1) + \dots + O(1)$$
$$O(1) = A+ O(1)$$
\end{enumerate}
\subsubsection{Funkcja liniowa}
\label{sec:org808352c}
$$F(N) = A \cdot N + B, A,B = const$$
\(\alpha = \text{tg } A\)
\begin{enumerate}
\item O(N)
\label{sec:org09d22ea}
$$O(N) = A \cdot O(N)$$
$$O(N) = O(N) + O(N) + \dots + O(N)$$
$$O(N) = O(1) \cdot O(N)$$
$$O(N) = N \cdot O(1)$$
\end{enumerate}
\subsubsection{Funkcja kwadratowa}
\label{sec:org2bc76db}
\(F(N)=A \cdot N^2 + B \cdot N + C\)
\begin{enumerate}
\item \(O(N^2)\)
\label{sec:org0e06d35}
$$O(N^2) = A \cdot O(N^{2})$$
$$O(N^2) = O(N^{2}) + O(N^{2}) + \dots + O(N^{2})$$
$$O(N^2) = O(N^{2}) + O(N)$$
$$O(N^2) = O(N^{2}) + O(1)$$
$$O(N^2) = N \cdot O(N)$$
$$O(N^2) = O(1) \cdot O(N^2)$$
$$O(N^2) = O(N) \cdot O(N)$$
\end{enumerate}
\subsubsection{Funkcja wielomianowa}
\label{sec:orge6962cd}
$$F(N) = A \cdot N^{B} + \cdots + X, A,B,\cdots,X = const $$
\subsubsection{Funkcja wykłatnicza}
\label{sec:org53647e7}
$$F(N) = A^{N} + B^{B_{1}} + \dots + X}$$
\subsubsection{Funkcja silnia wykładnicza}
\label{sec:org2025af4}
\(f(N) = N!\)
\section{Klasy}
\label{sec:orgceb8ed1}
\subsection{\(P, NP\)}
\label{sec:org6cb14f0}
Zadania klasy \(P\), są to zadania które są rozwiązywalne przez algorytm w czasie wielominaowym, przez deterministyczną maszynę Turinga.

\subsection{\(NP\)}
\label{sec:orgaf14442}
Są to zadnia dla których instnieją algorytmy które dają przybliżone rozwiązanie w czasie wielomianowym na niedetermistycznej maszynie Turinga.
\end{document}