% Created 2022-12-08 Thu 23:29
% Intended LaTeX compiler: pdflatex
\documentclass[11pt]{article}
\usepackage[utf8]{inputenc}
\usepackage[T1]{fontenc}
\usepackage{graphicx}
\usepackage{longtable}
\usepackage{wrapfig}
\usepackage{rotating}
\usepackage[normalem]{ulem}
\usepackage{amsmath}
\usepackage{amssymb}
\usepackage{capt-of}
\usepackage{hyperref}
\author{Rafał Grot}
\date{\today}
\title{Wyklad08}
\hypersetup{
 pdfauthor={Rafał Grot},
 pdftitle={Wyklad08},
 pdfkeywords={},
 pdfsubject={},
 pdfcreator={Emacs 28.2.50 (Org mode 9.6)}, 
 pdflang={English}}
\begin{document}

\maketitle
\tableofcontents


\section{Stack}
\label{sec:orgd7f99bb}
\begin{verbatim}
#include <iostream>
/*
template <typename T> class IStack {
public:
    virtual void push(const T& data);
    virtual T pop();
    virtual int getCount();
}; */
template <typename T> class Stack // : public IStack<T>
{
    struct Node
    {
        Node* next;
        T data;
    };
    Node* m_fp {nullptr};
    int m_count{};

public:
    void push(const T& data);
    T pop();
    int getCount();
    ~Stack();
    Stack();
};

template <typename T> void Stack<T>::push(const T& data) //O(1)
{
    m_fp = new Node{m_fp, data};
    ++m_count;
}

template <typename T> T Stack<T>::pop() //O(1)
{
    if (m_fp == nullptr)
        return T{};

    T data = m_fp->data;
    Node* toremove{m_fp};
    m_fp = m_fp->next;
    delete toremove;
    --m_count;
    return data;
}

template <typename T> int Stack<T>::getCount() // O(1)
{
    return m_count;
}

template <typename T> Stack<T>::~Stack() //O(N)
{
    while (m_fp != nullptr)
        pop();
}

template <typename T> Stack<T>::Stack() //O(1)
:m_fp{nullptr}, m_count{}
{}

int main()
{
    Stack<int>* stack = new Stack<int>;
    std::cout << stack->getCount() << '\n';  // 0
    stack->push(1); //1
    std::cout << stack->getCount() << '\n'; // 1
    stack->push(2); //21
    stack->push(3); //321
    std::cout << stack->getCount() << '\n'; //3
    std::cout << stack->pop() << '\n'; //21
    std::cout << stack->getCount() << '\n'; //2
    std::cout << stack->pop() << '\n'; //1
    std::cout << stack->pop() << '\n'; //0
    std::cout << "stack empty\n";
    std::cout << stack->pop() << '\n'; // returns default <T> object
}
\end{verbatim}

\begin{verbatim}
0
1
3
3
2
2
1
stack empty
0
\end{verbatim}
\section{Queue}
\label{sec:orgf805b4c}

\begin{verbatim}
#include <iostream>
/*
template <typename T> class IQueue {
public:
    virtual void enqueue(const T& data);
    virtual T dequeue();
    virtual int getCount();
}; */
template <typename T> class Queue // : public IQueue<T>
{
    struct Node
    {
        Node* next;
        T data;
    };
    Node* m_fp {nullptr};
    Node* m_lp {nullptr};
    int m_count{};

public:
    void enqueue(const T& data);
    T dequeue();
    int getCount();
    ~Queue();
    Queue();
};

template <typename T> void Queue<T>::enqueue(const T& data) //O(1)
{
    Node *newnode = new Node{nullptr, data};
    if(m_fp == nullptr)
        m_fp=newnode;
    if(m_lp != nullptr)
        m_lp->next = newnode;
    m_lp=newnode;
    ++m_count;
}

template <typename T> T Queue<T>::dequeue() //O(1)
{
    if (m_fp == nullptr)
        return T{};

    T data = m_fp->data;
    Node* toremove{m_fp};
    m_fp = m_fp->next;
    delete toremove;
    --m_count;
    return data;
}

template <typename T> int Queue<T>::getCount() // O(1)
{
    return m_count;
}

template <typename T> Queue<T>::~Queue() //O(N)
{
    while (m_fp != nullptr)
        dequeue();
}

template <typename T> Queue<T>::Queue() //O(1)
{}

int main()
{
    Queue<int>* queue = new Queue<int>;
    std::cout << queue->getCount() << '\n';  // 0
    queue->enqueue(1); //1
    std::cout << queue->getCount() << '\n'; // 1
    queue->enqueue(2); //12
    queue->enqueue(3); //123
    std::cout << queue->getCount() << '\n'; //3
    std::cout << queue->dequeue() << '\n'; //23
    std::cout << queue->getCount() << '\n'; //2
    std::cout << queue->dequeue() << '\n'; //3
    std::cout << queue->dequeue() << '\n'; //_
    std::cout << "queue empty\n";
    std::cout << queue->dequeue() << '\n'; // returns default <T> object
}
\end{verbatim}

\begin{verbatim}
0
1
3
1
2
2
3
queue empty
0
\end{verbatim}
\end{document}