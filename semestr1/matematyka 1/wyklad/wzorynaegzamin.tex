% Created 2023-02-12 Sun 13:23
% Intended LaTeX compiler: pdflatex
\documentclass[11pt]{article}
\usepackage[utf8]{inputenc}
\usepackage[T1]{fontenc}
\usepackage{graphicx}
\usepackage{longtable}
\usepackage{wrapfig}
\usepackage{rotating}
\usepackage[normalem]{ulem}
\usepackage{amsmath}
\usepackage{amssymb}
\usepackage{capt-of}
\usepackage{hyperref}
\usepackage[polish]{babel}
\usepackage[margin=3cm]{geometry}
\newgeometry{vmargin={5mm}, hmargin={20mm,20mm}}
\author{placeholder}
\date{\today}
\title{}
\hypersetup{
 pdfauthor={placeholder},
 pdftitle={},
 pdfkeywords={},
 pdfsubject={},
 pdfcreator={Emacs 30.0.50 (Org mode 9.6)}, 
 pdflang={Polish}}
\begin{document}

\begin{align*}
c \in \mathbb{R} && a \in \mathbb{R}
\end{align*}
\section{Wzory na pochodne wybranych funkcji}
\label{sec:orge585e9f}
\begin{latex}
\begin{align*}
  & c' = 0,
  \\ \left( x^a \right)' &= a x^{a - 1},
                         & \left( a^{x} \right)' &= a^{x} \ln a ,
                                                 & \left( e^{x} \right)' &= e^{x},
  \\  \left( \log_{a}x \right)'&= \frac{1}{x \cdot \ln a},
                         & \left( \ln x \right)' &= \frac{1}{x}
  \\ \left( \sin x \right)' &= \cos x,
                         &   \left( \cos x \right)' &= - \sin x,
                                                 & \left( \text{tg } x \right)' &= \frac{1}{\cos^{2} x},
  \\ \left( \text{ctg } x  \right)' &= \frac{-1}{\sin^{2} x},
  \\ \left( \arcsin x \right)' &= \frac{1}{\sqrt{1-x^{2}}},
                         & \left( \arccos x \right)' &= \frac{-1}{\sqrt{1-x^{2}}},
                                                 & \left( \text{arctg } x \right)' &= \frac{1}{1+x^{2}},
  \\ \left( \text{arcctg } x \right)' &= \frac{-1}{1+x^{2}},
  \\ \left( \sinh x \right)' &= \cosh x,
                         & \left( \cosh x \right)' &= \sinh x,
                                                 & \left( \text{tgh } x \right)' &= \frac{ 1 }{ \cosh^{2} x},
  \\ \left( \text{ctgh } x \right)' &= \frac{-1}{ \sinh^{2} x}
\end{align*}
\end{latex}
\section{Pochodna sumy, różnicy, iloczynu, ilorazu funkcji}
\label{sec:org1486b92}
\begin{align*}
  & \left( f(x) + g(x) \right)' = f'(x) + g'(x)\\
  & \left( c \cdot f(x) \right) ' = c \cdot f'(x),& c \text{ -- liczba }\\
  & \left( f(x) \cdot g(x) \right) ' = f'(x) \cdot g(x) + f(x) \cdot g'(x)\\
  & \left( \frac{f(x)}{g(x)} \right) ' = \frac{f'(x) \cdot g(x) - f(x) \cdot g'(x)}{g^{2}(x)}, & \text{o ile } g \neq 0
\end{align*}
\section{Pochodna funkcji złożonej}
\label{sec:org2637947}
Dana jest funkcja złozona \(y = (g^\circ w)(x)\) czyli \(y = g(w(x))\).
\begin{center}
\(w = w(x)\) - funkcja wewnętrzna, \qquad \(y = g(w)\) - funkcja zewnętrzna
\end{center}
\subsection{Wzory na pochodne funkcji złożonych}
\label{sec:org102e482}
\begin{align*}
  & c' = 0,
  \\ \left(w^{a}\right)'&= a w^{a-1} \cdot w',
                        & \left(a^{w} \right)' &= a^{w} \ln a \cdot w',
                                               & \left( e^{w} \right)' &= e^{w} \cdot w',
  \\ \left(\log_{a}w \right)' &= \frac{1}{w \cdot \ln a} \cdot w',
                        & \left( \ln w \right)' &= \frac{1}{w} \cdot w',
  \\ \left( \sin w \right)' &= (\cos w) \cdot w',
                        & \left( \cos w \right)' &= (- \sin w) \cdot w',
                                               & \left( \text{tg } w \right)' &= \frac{1}{\cos^{2} w} \cdot w' ,
  \\ \left(\text{ctg } w \right)' &= \frac{1}{\sin^{2} w} \cdot w',
  \\  \left( \arcsin w \right)' &= \frac{1}{\sqrt{1-w^{2}} \cdot w'}
                        & \left(\arccos w \right)' &=\frac{1}{\sqrt{1+w^{2}}} \cdot w'
                                               & \left( \text{arctg } w \right)' &= \frac{1}{1+w^{2}} \cdot w',
  \\ \left( \text{arcctg } w \right)' &= \frac{-1}{1+w^{2}} \cdot w',
  \\  \left( \sinh w \right )' &=  (\cosh w) \cdot w' ,
                        & \left( \cosh w \right )' &= (\sinh w) \cdot w',
                                               & \left( \text{tgh } w \right )' &= \frac{1}{\cosh^{2} w} \cdot w',
  \\ \left( \text{ctgh } w \right )' &= \frac{-1}{\sinh ^{2} w} \cdot w',
\end{align*}

\section{Całki Oznaczone}
\label{sec:orgabd32b8}
\textbf{Definicja całki oznaczonej z funkcji \(f(x) \ge 0\) w przedziale \(<a,b>\)}
\\\empty
Dla każdej z lcizb \(n = 1,2,3, \ldots\) postępujemy następująco:
\\\empty
Przedział \(<a,b>\) dzielmy na podprzedziały punktami \(x_0 , x_1, x_2, \ldots x_n\).
\\\empty
Punkty \(u_1, u_2 , \ldots , u_n\) nazywamy \textbf{punktami podziału}.
\\\empty
Długości kolejnych podprzdziałów \(<x_0, x_1>, <x_1, x_2>, \ldots , <x_{n-1}, x_n>\) oznaczamy przez \(\Delta x_1, \Delta x_2 , \ldots ,\Delta x_n\).
\\\empty
Największą z tych liczb nazywamy \textbf{punktami pośrednimi}.
\\\empty
Tworzymy sumę, zwaną \textbf{sumą całkową}
\\\empty
\displaystyle
\(\sigma_n = f(u_1) \cdot \Delta x_1 +
f(u_2) \cdot \Delta x_2 +
\ldots
f(u_2) \cdot \Delta x_2
= \sum_{k=1}^n f(u_k) \cdot \Delta x_k\)
\\\empty
Jeśli istnieje i jest skończona granica sum całkowych \(\sigma_n\) przy \(n \to \infty\) oraz gdy zachodzą poniższe założenia 1-3, to granice tę nazywamy całką oznaczoną funkcji f(x) w przedziale \(<a,b>\).
\\\empty
Oznaczamy ją symbolem \displaystyle\(\int_a^b f(x) dx\) .
\begin{enumerate}
\item Średnica podzału musić zmierzać do 0, gdy n zmierza do \(\infty\).
\item Granica nie może zależeć od wyboru punktów podziału \(x_0, x_1, x_2, \ldots ,x_n\) dla \(n = 1,2,3,\ldots\)
\item Granica nie może zależeć od wyboru punktów pośrednich \(u_0, u_1, u_2, \ldots ,u_n\) dla \(n = 1,2,3,\ldots\)
\end{enumerate}

\textbf{krótko:}
\(\displaystyle\lim_{n\to \infty} \sigma_n = \int_a^b f(x) dx\) jeżeli granica ta jest skończona i zachodzą założenia 1-3.
Jeżeli \(\int_{a}^{b} f(x) dx\) istnieje, to \(f(x)\) nazywa się funkcją \textbf{całkowalną} w przedziale \(<a,b>\).\\\empty
Funkcja ciągła w przedziale domkniętym jest w tym przedziale całkowalna.
\begin{align*}
  \int_{a}^{b}\left( f(x) + g(x) \right)dx &= \int_a^b f(x)dx + \int_a^b g(x)dx
  & \int_{a}^{b} \lambda f(x)dx &= \lambda \int_{a}^{b}f(x)dx, \lambda \in \mathbb{R}
\\ \int_a^b f(x)dx &= - \int_b^a f(x)dx
\end{align*}
\subsection{Właściowści}
\label{sec:org9f24b4e}
Załóżmy, że funkcjie f(x), g(x) to funkcjie całkowalne w przedziale \(<a,b>\)
\begin{enumerate}
\item \(\displaystyle\int_a^b( f(x) + g(x) )dx
   = \int_a^b f(x) dx + \int_a^b g(x)dx\)
\item \(\displaystyle \int_a^b \lambda f(x) dx = \lambda \int_a^b f(x)dx, \lambda \in \mathbb{R}\)
\item \(\displaystyle \int_a^b f(x) dx = - \int_b^a f(x) dx\).
\item Niech \(c \in <a,b>\) wtedy \(\displaystyle \int_a^b f(x)dx = \int_a^c f(x)dx + \int_c^b+f(x)dx\).
\item Niech \(f(x) \le g(x)\) w \(<a, b>\) wtedy \(\displaystyle \int_a^b f(x)dx \le \int_a^b g(x)dx\).
\end{enumerate}
\subsection{Twierdzenie o wratości średniej}
\label{sec:orgac3b3dc}
Dla dowolonej funkcji ciągłej f(x) w przedziale \(<a,b>\) istnieje taka liczba \(h \in <a,b>\), że
$$\int_a^b f(x)dx = (b-a)f(h)$$
\newpage
\section{Całki nieoznaczone}
\label{sec:orgacf6d0d}
Funkcja \(F\) jest \textbf{funkcją pierwotną} funkcji \(f\) na przedziale \(I\) , jeżeli \(F'(x) = f(x)\) dla każdego \(x \in I\).
\subsection{Twierdzenie (warunek wystrczający instnienia funkcji pierwotnej)}
\label{sec:orgc11bd7e}
Jeżeli funkcja jest ciagłą na przedzale to, ma funkcję pierwotną na tym przedziale.
\subsection{Definicja}
\label{sec:orgb0671a1}
\textbf{Całkę nieoznaczoną} funkcji \(f\) zapisujemy w postaci \(\int f(x)dx\) i definiujemy następująco:
$$\int f(x) dx = F(x) + c, \text{ gdy }F'(x) = f(x)$$
\(c\) -- stała całkowania
\subsection{Całki nioznaczone pewnych funkcji elementarnych}
\label{sec:orge8ebcaa}
\begin{enumerate}
\item \(\displaystyle\int 0 dx = c , x \in \mathbb{R}\)
\item \(\displaystyle\int x^{a}dx = \frac{x^{a+1}}{a + 1} +c\) dla \(a \ne -1\), zakres \(x\) zależy od \(a\)
\item \(\displaystyle\int x^{-1}dx = \int \frac{1}{x}dx = \ln |x| + c\) dla \(x \in ( - \infty,0 )\) lub \(x \in (0, + \infty)\)
\item \(\displaystyle\int a^x dx = \frac{a^x}{\ln a} + c\) dla \(a>0\) i \(a \ne 1, x \in \mathbb{R}\)
\item \(\displaystyle\int e^{x} dx = e^x + c\) dla \(x \in \mathbb{R}\)
\item \(\displaystyle\int \sin x dx = -\cos x +c\) dla \(x \in \mathbb{R}\)
\item \(\displaystyle\int \cos x dx = \sin x + c\) dla \(x \in \mathbb{R}\)
\item \(\displaystyle\int \frac{1}{\cos^2 x}dx = \tan x + c\) dla \(\displaystyle x \in \left( - \frac{\pi}{2} + k \pi, \frac{\pi}{2}+k \pi \right)\) gdzie \(k \in \mathbb{Z}\)
\item \(\displaystyle\int \frac{1}{\sin^2 x}dx = - \cot x+c\) dla \(x \in (k\pi, (k+1)\pi\) gdzie \(k \in \mathbb{Z}\)
\item \(\displaystyle\int \frac{dx}{1 + x^2} = \arctan x + c\) dla \(x \in \mathbb{R}\)
\item \(\displaystyle\int \frac{dx}{\sqrt{1-x^2}} = \arcsin x + c\) dla \(x \in (-1, 1)\)
\item \(\displaystyle\int \sinh dx = \cosh x + c\) dla \(x \in \mathbb{R}\)
\item \(\displaystyle\int \cosh x = \sinh x + c\) dla \(x \in \mathbb{R}\)
\item \(\displaystyle\int \frac{1}{\cosh^2 x} dx = \tanh x + c\) dla \(x \in \mathbb{R}\)
\item \(\displaystyle\int \frac{1}{\sinh^2 x} dx = - \coth x + c\) dla \(x \in ( - \infty , 0 )\) lub \(x \in (0 , + \infty)\)
\item \(\displaystyle\int \sin x dx = -\cos x + c\)
\item \(\displaystyle\int \frac{1}{\sin^2 x}dx = -\cot x + c\)
\item \(\displaystyle\int \frac{1}{\sqrt{1 -x ^2}}dx = \arcsin x + c\)
\end{enumerate}
\subsection{Twierdzenie}
\label{sec:orgd5e66c2}
\begin{enumerate}
\item \(\displaystyle\int \left( f(x) + g(x) \right) dx = \int f(x)dx + \int g(x)dx\)
\item \(\displaystyle\int (c f(x))dx = c \int f(x) dx, c \in \mathbb{R}\)
\item \(\displaystyle \left( \int f(x) dx \right)' = f(x)\)
\item \(\displaystyle \int f'(dx) = f(x) +c\), gdzie \(c \in \mathbb{R}\)
\end{enumerate}
\end{document}