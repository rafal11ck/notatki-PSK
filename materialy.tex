% Created 2022-10-05 Wed 10:42
% Intended LaTeX compiler: pdflatex
\documentclass[11pt]{article}
\usepackage[utf8]{inputenc}
\usepackage[T1]{fontenc}
\usepackage{graphicx}
\usepackage{longtable}
\usepackage{wrapfig}
\usepackage{rotating}
\usepackage[normalem]{ulem}
\usepackage{amsmath}
\usepackage{amssymb}
\usepackage{capt-of}
\usepackage{hyperref}
\author{Rafał Grot}
\date{\today}
\title{Źródła materiałów 1ID12B}
\hypersetup{
 pdfauthor={Rafał Grot},
 pdftitle={Źródła materiałów 1ID12B},
 pdfkeywords={},
 pdfsubject={},
 pdfcreator={Emacs 28.2 (Org mode 9.6)}, 
 pdflang={English}}
\begin{document}

\maketitle
\tableofcontents


\section{Arechitektura systemów komputerowych}
\label{sec:org96b8ad4}
\begin{description}
\item[{wykłady}] \url{https://hector.tu.kielce.pl/przedmioty/ask-wyklady.html}
\end{description}
\section{Podstawy porgramowania 1}
\label{sec:orgb11f382}
\begin{description}
\item[{wykłady}] \url{https://achilles.tu.kielce.pl/portal/Members/84df831b59534bdc88bef09b15e73c99/podstawy-programowania-1}
\item[{laboratioria}] \url{https://weaii-moodle.tu.kielce.pl/enrol/index.php?id=149} trzeba się logować oraz kod \textbf{\texttt{kubit\_12B}}
\end{description}
\section{Teoria układóœ logicznych}
\label{sec:orgefc0a80}
\begin{description}
\item[{laboratoria}] \url{https://weaii-moodle.tu.kielce.pl/course/view.php?id=304} potrzebny kod \textbf{tul2022}
\end{description}
\section{Podstawy elektorniki}
\label{sec:org920846f}
\begin{description}
\item[{laboratioria}] ???
\end{description}
\end{document}
