% Created 2022-10-11 Tue 15:41
% Intended LaTeX compiler: pdflatex
\documentclass[11pt]{article}
\usepackage[utf8]{inputenc}
\usepackage[T1]{fontenc}
\usepackage{graphicx}
\usepackage{longtable}
\usepackage{wrapfig}
\usepackage{rotating}
\usepackage[normalem]{ulem}
\usepackage{amsmath}
\usepackage{amssymb}
\usepackage{capt-of}
\usepackage{hyperref}
\author{Rafał Grot}
\date{\today}
\title{Wyklad02}
\hypersetup{
 pdfauthor={Rafał Grot},
 pdftitle={Wyklad02},
 pdfkeywords={},
 pdfsubject={},
 pdfcreator={Emacs 28.2 (Org mode 9.6)}, 
 pdflang={English}}
\begin{document}

\maketitle
\tableofcontents

\section{Cykl pracy komputrea}
\label{sec:org34c4d30}
\subsection{Uruchomienie komputera}
\label{sec:orgeb1b0d8}
\subsubsection{Włączenie zasilania}
\label{sec:orgf562129}
\begin{enumerate}
\item przejście w stan wysokiej impedancji
\label{sec:orgc360e23}
\item moduły wykonują testy wewnętrzne
\label{sec:org1ade9c5}
\item mikroprocesor wystawia na magistralę adresową wartość FFFF:0 (zapis segment:offset)
\label{sec:org5ed411d}
\item mikroprocesor pobiera zawartość zaadresowaniej pamięci i rozpoczyna wykonanie programu
\label{sec:org35132eb}
\begin{enumerate}
\item BIOS
\label{sec:org43cb3e4}
Umieszczony w pamięci nieulotnej, zawiera podprogram POST
\begin{enumerate}
\item POST
\label{sec:orga429f8f}
Jest pierwszą prcedurą uruchomioną bo BIOS
\begin{enumerate}
\item Weryfikuje rejsetry CPU :: zapsuje wartości w rejestrach, każda jest przygotowana dla danej architekttury procesora
\label{sec:org819f2d9}
\item Weryfikuje integralność BIOSu
\label{sec:orgcb404cf}
Liczenie sumy kontrolnej.
\item Werifikuje komponenty komputera
\label{sec:org2d9c6e1}
\begin{enumerate}
\item DMA
\label{sec:orgb570be2}
\item Timer
\label{sec:orgaf46ed6}
\item Kontroler przerwań
\label{sec:orgad6f1ea}
\item Sprawdzenie i weryfikacja pamięci operacyjnej.
\label{sec:org59d7996}
\end{enumerate}
\item Zainicjowanie pamięci
\label{sec:org13eed5b}
\begin{enumerate}
\item katalogwanie
\label{sec:org2787c60}
\begin{itemize}
\item magistrali komputera
\end{itemize}
\end{enumerate}
\end{enumerate}
\end{enumerate}
\end{enumerate}
\end{enumerate}
\subsubsection{Ładowanie systemu operacyjnego}
\label{sec:org2d2d93a}
Z pamięci masowej, blokowej
\begin{itemize}
\item katalogwanie urządzeń we/wy
\item katalogowanie urządzeń blokowych
\item Ładowanie sterowników
\item zainicjowanie systemu plików
\end{itemize}
\subsubsection{Inicjacja systemu operacyjnego}
\label{sec:org5179268}
\begin{enumerate}
\item uruchamianie usług
\label{sec:org4fcc2fe}
usługa to np serwer wydruku.
\item ładowanie aplikacji w tle
\label{sec:orgd4430c5}
Np sterownik wspomagająćy działanie klawiatury lub myszy.
\item Organizacja pamięci
\label{sec:orgea28136}
\item Uruchomienie powłoki
\label{sec:orgf29f8ef}
\end{enumerate}
\subsubsection{Faza aplikacji}
\label{sec:org3ee07a6}
\begin{enumerate}
\item urzuchomienie aplikacji
\item zakończenie aplikacji
\item oczekiwanie na polecenia
\item od nowa aż do zamknięcia systemu
\end{enumerate}
\subsubsection{zamykanie systemu operacyjnego}
\label{sec:org9cf0c15}
\begin{enumerate}
\item zapisanie danych
\label{sec:org3bce0c8}
\item zamknięcie plikóœ
\label{sec:org678d203}
\item zwolnienie zasobów
\label{sec:org34f4641}
\item wyłączenie komputera
\label{sec:orgc5a4b7c}
\end{enumerate}
\section{Konstrukcja i zasada działąnia mikroprocesora}
\label{sec:org6381023}
\subsection{mikroprocesor}
\label{sec:org8eec540}
Jest ukłądem ctfrowym skewencyjnym, wykonujączym polecenai(instrukcje). CPU jest jedonstka obliczeniwą komputera.
Konstrukcyjnie każdy procesor jest układem FSM wykonanym zgodnie z modelem RTL.
\subsubsection{Typy}
\label{sec:orgdf2633a}
\begin{enumerate}
\item SISD
\label{sec:org3f52880}
Single instruction single data.
\item SIMD
\label{sec:org407569c}
Pojedyńczy strumień instrukcji i wiele strumieni danych.
np: MMX+.
\item MISD
\label{sec:org654e264}
Wiele strumni instrukcji, jeden strumień danych.
\item MIMD
\label{sec:orgb266fd8}
Wiele strumieni instukcji, wiele strumieni danych.
\end{enumerate}
\subsubsection{Rodzaje}
\label{sec:org1621382}
\begin{enumerate}
\item CISC (Complex Insturcion Set Computer)
\label{sec:orgc1705bc}
Mnożenie, dzielenie.
\item RISC (Reduced Insturcion Set Computer)
\label{sec:orgb6ab160}
np: architektura ARM.
Potrzeba mniej zasobóœ sprzętowych żeby zrealzować układ, więc mniejsze zurzycie prądu.
\item VLIW (Very long instruction word)
\label{sec:org15afcb5}
np: Intel Itanium
\end{enumerate}
\subsubsection{Moduły}
\label{sec:org1cb3ba7}
\begin{enumerate}
\item Ścieżka danych
\label{sec:org2eea4af}
\begin{itemize}
\item blok rejstróœ ogólnego przeznacznia
\item pamięci podręczne pierwszego poziomu
\item rejestry adresowe
\item pamięć stronnicowania i translaci adresóœ TLB
\item układ arytmetyczno-logiczny
\end{itemize}
\item moduły kontrolera
\label{sec:orgcfb1a41}
\begin{itemize}
\item sterownik magistal
\item układy sterujące
\item układ adresowy
\item blok pobierania rozkazów
\item[{dekoder instrukcji}] mówi jak instukcja zostanie wykonana
\end{itemize}
\end{enumerate}


\subsection{Architektury}
\label{sec:org1fc361a}
\subsubsection{von Neumana}
\label{sec:org2a0eb6c}
cechy:
\begin{itemize}
\item posadanie skończnej, w pełnu fukcjonalnie listy rozkazów.
\item posadanie możliwośći wprdazanai programóœ oraz przchowywania ich w pamieći.
\item dane i rozkazy powwiny być swobodnie dostępnie.
\item przetwarzaine informacji następuje na skutek sekwencyjnego odczytywania instrukcji z pamięci i wykonywania ich przez procesor komputera.
\end{itemize}
\subsubsection{Harwardzka}
\label{sec:org6c3af21}
Starsza od von Neumana.
cechy:
\begin{itemize}
\item rozdzielenie pamięci danych i pamięci programu (podział logiczny).
\item instrukcjie posiadaja prostą konstrukcję, nie występuje mikrokod.
\end{itemize}
\subsection{Rodziny procoesorów}
\label{sec:org6419fa4}
\subsubsection{X86/IA-32}
\label{sec:orgbedf7cf}
\begin{enumerate}
\item cechy:
\label{sec:orgb338cb5}
\begin{itemize}
\item rodzaju CISC, typu SISD
\item posiada 16/32 bitową archtektórę
\item zapisuje słowa metodą little-endian
\item największa wartość jest wielkości 16,32,64 bitów(zależnie od rodziny)
\item mogącym współpracować z koprocesorem
\item potrafiąćym zaadresować 1MB pamięci RAM (4FB procesor 386+)
\item posiada wiele trybów adresowania pamięci
\end{itemize}
\item tryby pracy
\label{sec:org8e1a9b5}
\begin{enumerate}
\item rzeczywisty 8086
\label{sec:org3d8cf58}
\begin{itemize}
\item może zadresować 1MB RAM
\item nie ma możliwości ochrony pamięci, zarządzania zadaniami, wątkami
\item nie są dostępnie roszerzone instrukcje
\item dostęp do urządzeń jest możliwy przez wywałania funkcji BIOSu
\end{itemize}
\item chronionym
\label{sec:org5ae809d}
\begin{itemize}
\item dostępna jest cała przestrzeń adresowa.
\item pamięc i zadania są chronione: segmenty definiuje się w tabliczach GDT i LDT
\item dostępna jest pamięć wirtualna oraz stronnciowanie
\item dostępna jest wieleozadaniowość (wielowątkowość)
\item[{dostępny jest tryb ``virtual 8086 mode''}] procesor emuluje jeden ze swoich poprzedników
\end{itemize}
\item rejestry
\label{sec:orgff45d9c}
\begin{enumerate}
\item ogólnego przeznaczenia
\label{sec:org447acce}
\begin{enumerate}
\item akumulator
\label{sec:orgc053727}
AL,AH,AX,EAX<RAX
\item ideksowo bazowy
\label{sec:orgdbb11ec}
BL,BH,BX,EBX,RBX
\item licznik
\label{sec:org9b95a90}
CL,CH,CX,ECXRCX
\item roszerzająćy akumulator
\label{sec:orgbef8cae}
DL,DH,DX
\item indeks źródła SI,ESI,RSI
\label{sec:org48aaf42}
\item ideks przeznaczenia
\label{sec:org7404185}
DI,EDI,RDI
\item wskaźnik stosu
\label{sec:org90ed1aa}
SP,ESP,RSP
\item bazowy stosu (ramki stosu)
\label{sec:orgd22785a}
BP,EBP,RBP
\item licznika programu
\label{sec:org22051d6}
IP,EIP,RIP
\end{enumerate}
\item segmentowe
\label{sec:org68813b1}
\begin{itemize}
\item programu CS
\item stosu SS
\item danych DS,ES,FS,GS
\end{itemize}
\item stanu (flags)
\label{sec:orgef44651}
\item kontrolne (Crx)
\label{sec:org076bf8c}
\item debuggera
\label{sec:org14130e3}
\end{enumerate}
\end{enumerate}
\item zarządzanie panięcią
\label{sec:org2bfdcad}
\begin{enumerate}
\item tryby adresowania
\label{sec:orgfe83394}
\begin{itemize}
\item natychmiastowe
\item rejestrowe
\item bezpośrednie
\item pośrednie
\item bazowe
\item indeksowe
\item bazowo-ideksowe
\end{itemize}
\item mechanizym ochrony
\label{sec:org68dd66c}
Typu sektor:przemieszczenie bazujące na deskryptorach segmentóœ globlanych GDT i lokalnych LDT.
\item obsługa stronicowania
\label{sec:orgde4aa5b}
wykorzystuje tablicę TLB do odwzorwania nieciągłego obszaru pamięci fizycznej w ciągłe obszary pamięci logicznej (segmentów).
\item pamięć wirtualna
\label{sec:org2d563c3}
pozwala na wymianę stron pamięci RAM z pamięcią masowoą w trakcie odwołania do segmentów pamięci.
\end{enumerate}
\end{enumerate}
\subsubsection{IA-64}
\label{sec:orgee2039a}
Zostałą apracowana przez firmy intel oraz Hewlett-Packard.
\begin{enumerate}
\item cechy:
\label{sec:org807ba1f}
\begin{itemize}
\item rodzaju CISC/RISC typu MIMD (SIMD)
\item posadającym 128 rejestrów ogólnego przeznaczenia (16 typu integer, 96 do dyspozycji aplikacji (alokowalnych))
\item super-skalarny
\item posiada zaawansowany mechanizm potokowy.
\item posiada możliwość wykonania instrukcji w innej kolejności (out-of-order execution)
\item posiada mechanizm spekulatywnego wykonywania rozkazów
\item potrafi wyknać 12 instrukcji w jednym cyklu zegara (Itanium 9500 series)
\item posiada zaawansowane mechanizmy wirtualizacji
\item wyknującym instrukcje w trybie warunkowym
\end{itemize}
\item EPIC / VLIW
\label{sec:org4b5e1f1}
Archiitektura IA-64 jest odmianą modelu EPIC ()ang. Explicitly Paraller Instruction Computing), będącego rodzajem modelu MIMD.
\begin{enumerate}
\item cechy:
\label{sec:org2aa273a}
\begin{itemize}
\item zazwyczaj są rodzaju RISC lub o podobnym modelu oblcizeniowym
\item wielordzeniwoość
\end{itemize}
\end{enumerate}
\end{enumerate}
\subsubsection{ARM}
\label{sec:orgf25b9f6}
RISK
\end{document}
