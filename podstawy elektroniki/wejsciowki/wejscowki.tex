% Created 2022-12-22 Thu 12:19
% Intended LaTeX compiler: pdflatex
\documentclass[11pt]{article}
\usepackage[utf8]{inputenc}
\usepackage[T1]{fontenc}
\usepackage{graphicx}
\usepackage{longtable}
\usepackage{wrapfig}
\usepackage{rotating}
\usepackage[normalem]{ulem}
\usepackage{amsmath}
\usepackage{amssymb}
\usepackage{capt-of}
\usepackage{hyperref}
\author{Rafał Grot}
\date{\today}
\title{Wejscowki}
\hypersetup{
 pdfauthor={Rafał Grot},
 pdftitle={Wejscowki},
 pdfkeywords={},
 pdfsubject={},
 pdfcreator={Emacs 30.0.50 (Org mode 9.6)}, 
 pdflang={English}}
\begin{document}

\maketitle
\tableofcontents

\section{Tranzystor bipolarny}
\label{sec:org8563ed5}
\subsection{poprawa 1}
\label{sec:orgb7d5e0e}
\begin{enumerate}
\item Układy pracy tranzystora bioplarnego. narysować i podpisać
\item charakterystki w układzie wspólnego emitera
\end{enumerate}
\section{Tranzystor polowy}
\label{sec:org609d191}
\subsection{poprawa 1}
\label{sec:orgd8b6470}
\begin{enumerate}
\item charakterystki tranzystora polowego
\item transkonduktacnja wzór i wyznaczanie z charakterysyk
\end{enumerate}
\end{document}