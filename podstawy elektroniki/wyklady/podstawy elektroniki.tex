% Created 2022-10-15 Sat 16:36
% Intended LaTeX compiler: pdflatex
\documentclass[11pt]{article}
\usepackage[utf8]{inputenc}
\usepackage[T1]{fontenc}
\usepackage{graphicx}
\usepackage{longtable}
\usepackage{wrapfig}
\usepackage{rotating}
\usepackage[normalem]{ulem}
\usepackage{amsmath}
\usepackage{amssymb}
\usepackage{capt-of}
\usepackage{hyperref}
\author{Rafał Grot}
\date{\today}
\title{Podstway Elektroniki 01, instrukcje do lab są bardzo bezużyteczne bo zkaładają że coś wiem.}
\hypersetup{
 pdfauthor={Rafał Grot},
 pdftitle={Podstway Elektroniki 01, instrukcje do lab są bardzo bezużyteczne bo zkaładają że coś wiem.},
 pdfkeywords={},
 pdfsubject={},
 pdfcreator={Emacs 28.2 (Org mode 9.6)}, 
 pdflang={English}}
\begin{document}

\maketitle
\tableofcontents


\section{Półprzewodniki :: \url{https://pl.wikipedia.org/wiki/P\%C3\%B3\%C5\%82przewodniki}}
\label{sec:orgf0e440d}
\subsection{Samoistne}
\label{sec:orgaf0e739}
Są kijowe bo mają mało ładunków swobodnych(duża rezystywność).
Dlatego ich się nie stosuje.
\subsection{Domieszkowe (niesamoistne)}
\label{sec:orgd7ea2db}
Domieszkowanie polega na wprowadzeniu i aktywowaniu atomoów domieszk do struktury kryształu. Domieszki są atomami pierwiastków niewychodzących w skład połprzeodnika samiostnego.
\subsubsection{Półprzewodniki typu n}
\label{sec:orgfb6cdbe}
\begin{itemize}
\item Powstają przez wyprowadzenie domieszki mającej nadmiar elketronów(w stosunku do połprzewodnika samoistnego). Domieszkę nazywa się \textbf{donorową} (\emph{``oddaje elektron''}).
\item W takim półprzewodniku powstaje dodatkowy poziom energetyczny (\textbf{poziom donorowy}) w obszarze pasma wzbroninoego tuż pod poziomem przewodnictwa lub w nim.
\item Nadmiar elektronów jest uwalniany do pasma przewodzenia prądu w postaci elektronów swobodnych zdolnych do przewdzenia prądu. \textbf{Przewodnictwo elektrnowe lub przewodnictwo typu n (z ang.negative, ujemny)}
\end{itemize}
\subsubsection{Półprzewodniki typu p}
\label{sec:orgc4ffbca}
\begin{itemize}
\item Powstają przez wyprowadzenie domieszki dającej niedobór elektronów (w stosunku do półprzewdonika samiostnego). Domieszkę nazywa się \textbf{akceptorową} (\emph{przyjmuje elektron})
\item W takim półprzewodniku powstaje dodatkowy poziom energetyczny (\textbf{poziom akceptorowy}) w obszarze pasma wzbroninoego niewiele nad poziomem walencyjnym lub na nim.
\end{itemize}
\section{A potem znalazłem to i nie potrzebnie się męczyłem}
\label{sec:org48eb157}
\url{https://www.youtube.com/watch?v=NA7M-h0pxns\&list=PLpM1FRFG\_H-gvpcqDR9XeAEJ9a8eCTdxo}
\end{document}
