% Created 2023-01-10 Tue 18:21
% Intended LaTeX compiler: pdflatex
\documentclass[11pt]{article}
\usepackage[utf8]{inputenc}
\usepackage[T1]{fontenc}
\usepackage{graphicx}
\usepackage{longtable}
\usepackage{wrapfig}
\usepackage{rotating}
\usepackage[normalem]{ulem}
\usepackage{amsmath}
\usepackage{amssymb}
\usepackage{capt-of}
\usepackage{hyperref}
\renewcommand\maketitle{}
\usepackage[polish]{babel}
\usepackage[margin=3cm]{geometry}
\author{Rafał Grot}
\date{\today}
\title{}
\hypersetup{
 pdfauthor={Rafał Grot},
 pdftitle={},
 pdfkeywords={},
 pdfsubject={},
 pdfcreator={Emacs 30.0.50 (Org mode 9.6)}, 
 pdflang={Polish}}
\begin{document}


\section{Wzory na pochodne wybranych funkcji}
\label{sec:org771bcc7}
\begin{align*}
c \in \mathbb{R} && \alpha \in \mathbb{R}
\end{align*}
\begin{align*}
  & c' = 0,
  \\ \left( x^a \right)' &= a x^{a - 1},
                              & \left( a^{x} \right)' &= a^{x} \ln a ,
                                                      & \left( e^{x} \right)' &= e^{x},
  \\  \left( \log_{a}x \right)'&= \frac{1}{x \cdot \ln a},
                              & \left( \ln x \right)' &= \frac{1}{x}
                                                      & \left( \sin x \right)' &= \cos x,
  \\   \left( \cos x \right)' &= - \sin x,
                              & \left( \text{tg } x \right)' &= \frac{1}{\cos^{2} x},
                                                      & \left( \text{ctg } x  \right)' &= \frac{-1}{\sin^{2} x},
  \\
  \left( \arcsin x \right)' &= \frac{1}{\sqrt{1-x^{2}}},
                              & \left( \arccos x \right)' &= \frac{-1}{\sqrt{1-x^{2}}},
                                                      & \left( \arctan \right)' &= \frac{1}{1+x^{2}},\\
  \left( \text{arcctg } x \right)' &= \frac{-1}{1+x^{2}},
                              & \left( \sinh x \right)' &= \cosh x,
                                                      & \left( \cosh x \right)' &= \sinh x,
  \\ \left( \text{tgh } x \right)' &= \frac{ 1 }{ \cosh^{2} x},
                              &  \left( \text{ctgh } x \right)' &= \frac{-1}{ \sinh^{2} x}
\end{align*}
\section{Pochodna sumy, różnicy, iloczynu, ilorazu funkcji}
\label{sec:orgd89263c}

\begin{align}
  & \left( f(x) + g(x) \right)' = f'(x) + g'(x)\\
  & \left( c \cdot f(x) \right) ' = c \cdot f'(x),& c \text{ -- liczba }\\
  & \left( f(x) \cdot g(x) \right) ' = f'(x) \cdot g(x) + f(x) \cdot g'(x)\\
  & \left( \frac{f(x)}{g(x)} \right) ' = \frac{f'(x) \cdot g(x) - f(x) \cdot g'(x)}{g^{2}(x)}, & \text{o ile } g \neq 0\\
\end{align}
\newpage
\section{Pochodna funkcji złożonej}
\label{sec:orga2edb71}
Dana jest funkcja złozona \(y = (g^\circ w)(x)\) czyli \(y = g(w(x))\).

\(w = w(x)\) - funkcja wewnętrzna

\(y = g(w)\) - funkcja zewnętrzna
\subsection{Wzory na pochodne funkcji złożonych}
\label{sec:orgc1bfc69}
\begin{align*}
c \in \mathbb{R} && \alpha \in \mathbb{R}
\end{align*}
\begin{align*}
  & c' = 0,
  \\ \left(w^{a}\right)'&= a w^{a-1} \cdot w',
                        & \left(a^{w} \right)' &= a  \cdot w^{a-1} \cdot w',
                                               & \left( e^{w} \right)' &= e^{w} \cdot w',
  \\ \left(\log_{a}w \right)' &= \frac{1}{w \cdot \ln a} \cdot w',
                        & \left( \ln \right)' &= \frac{1}{w} \cdot w',
  \\ \left( \sin w \right)' &= (\cos w) \cdot w',
                        & \left( \cos w \right)' &= (- \sin w) \cdot w',
                                               & \left(\text{ctg } w \right)' &= \frac{1}{\sin^{2} w} \cdot w',
  \\ \left( \text{tg } w \right)' &= \frac{1}{\cos^{2} w} \cdot w' ,
                        & \left( \arcsin w \right)' &= \frac{1}{\sqrt{1-w^{2}} \cdot w'}
                                               & \left( \text{arcctg } w \right)' &= \frac{1}{1+w^{2}} \cdot w',
  \\ \left( \sinh w \right )' &=  (\cosh w) \ cdot w' ,
                        & \left( \cosh w \right )' &= (\sinh w) \cdot w',
                                               & \left( \text{tgh } w \right )' &= \frac{1}{\cosh^{2} w} \cdot w',
  \\ \left( \text{ctgh } w \right ) &= \frac{-1}{\cosh ^{2} w} \cdot w',
\end{align*}
\end{document}