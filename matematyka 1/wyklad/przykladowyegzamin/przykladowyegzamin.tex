% Created 2023-02-05 Sun 17:46
% Intended LaTeX compiler: pdflatex
\documentclass[11pt]{article}
\usepackage[utf8]{inputenc}
\usepackage[T1]{fontenc}
\usepackage{graphicx}
\usepackage{longtable}
\usepackage{wrapfig}
\usepackage{rotating}
\usepackage[normalem]{ulem}
\usepackage{amsmath}
\usepackage{amssymb}
\usepackage{capt-of}
\usepackage{hyperref}
\author{placeholder}
\date{\today}
\title{Przykladowyegzamin}
\hypersetup{
 pdfauthor={placeholder},
 pdftitle={Przykladowyegzamin},
 pdfkeywords={},
 pdfsubject={},
 pdfcreator={Emacs 30.0.50 (Org mode 9.6)}, 
 pdflang={English}}
\begin{document}

\maketitle
\tableofcontents


\section{{\bfseries\sffamily DONE} Zad 1}
\label{sec:orgcbb3986}
\begin{align*}
\Im \left(\frac{1+3i}{3-2i} + i^{3} + 5\right)
 &=\Im \left(\frac{1+3i}{3-2i} + \frac{i^{3}(3-2i)}{3-2i} + \frac{5(3-2i)}{3-2i}\right)\\
 &= \Im \left(\frac{1+3i + 3i^3 - 2 i^4 + 15 - 10i}{3-2i}\right)\\
 &= \Im \left(\frac{16 - 7i + 3i^{3} -2i^{4}}{3-2i}\right)\\
 &= \Im \left(\frac{14 - 10i}{3-2i}\right)\\
 &= \Im \left(\frac{14 - 10i}{3-2i} \cdot \frac{3+2i}{3+2i}\right)\\
 &= \Im \left(\frac{42 + 28i - 30i + 20}{9 + 4}\right)\\
 &= \Im \left(\frac{62 - 2i }{13}\right)\\
 &= \frac{-2}{13}
\end{align*}
\section{{\bfseries\sffamily DONE} Zad 2}
\label{sec:org973fd11}
\begin{align*}
  \frac{ { (3 - 3i)}^{14} }
  { { (-1+i\sqrt{3}) }^{11} }
  &= \frac{z^{14}}{w^{11}}
\end{align*}
\subsection{\(z\)}
\label{sec:org282dc4f}
$$\sin(\varphi_z) = \frac{-3}{3\sqrt{2}}
 = \frac{-1}{\sqrt{2}}
 = \frac{-\sqrt{2}}{2} \to \varphi_z = \frac{7}{4}\pi$$

\begin{align*}
  z^{14} &= {(3 - 3i)}^{14}\\
  &= {(3-3i)}^{14}\\
  &= {(3\sqrt{2})}^{14}(\cos 14 \varphi + i \sin 14 \varphi)\\
  &= {(3\sqrt{2})}^{14} \left(\cos \left(14 \cdot \frac{7}{4} \pi \right) + i \sin \left(14 \cdot \frac{7}{4} \pi \right) \right)\\
  &= {(3\sqrt{2})}^{14} \left( \cos \left ( \frac{49}{2} \pi \right) + i \sin \left(\frac{49}{2} \pi \right) \right)\\
  &= {(3\sqrt{2})}^{14} \left( \cos \left ( \frac{1}{2} \pi \right) + i \sin \left(\frac{1}{2} \pi \right) \right)\\
  &= {(3\sqrt{2})}^{14} ( 0 + i 1 )\\
  &= {(3\sqrt{2})}^{14}i
\end{align*}
\subsection{\(w\)}
\label{sec:org5b83ee3}
$$\sin(\varphi_w) = \frac{\sqrt{3}}{\sqrt{4}} = \frac{\sqrt{3}}{2}
\to \varphi_w = \frac{2}{3} \pi$$

\begin{align*}
w^{11} &= 2^{11} \left( \cos \left(11 \cdot \frac{2}{3} \pi \right)
+ i \sin \left( 11 \cdot \frac{2}{3} \pi \right) \right)\\
&= 2^{11} \left( -\cos \frac{\pi}{3}
- i \sin \frac{\pi}{3} \right)\\
&= 2^{11} \left(- \frac{1}{2} - i \frac{\sqrt{3}}{2} \right)\\
&= 2^{10} \left(-1 - i \sqrt{3} \right)
\end{align*}
\subsection{Podstawiamy}
\label{sec:orge1521f7}
\begin{align*}
\frac{ { (3 - 3i)}^{14} }
{ { (-1+i\sqrt{3}) }^{11} }
&= \frac{z^{14}}{w^{11}}\\
&=\frac{(3\sqrt{2})^{14} i }
{2^{10}(-1 -i\sqrt{3})}\\
&=\frac{ ((3\sqrt{2})^{14} i)(-1 + i\sqrt{3}) }
{2^{10}(-1 -i\sqrt{3})(-1 + i\sqrt{3})}\\
&=\frac{ ((3\sqrt{2})^{14} i)(-1 + i\sqrt{3}) }
{2^{10}(-2)}\\
&=\frac{ ((3\sqrt{2})^{14} i)(-1 + i\sqrt{3}) }
{-2^{11}}
\end{align*}
\section{{\bfseries\sffamily DONE} Zad 3}
\label{sec:orgd0e2a50}
Wyznacznik macierzy głownej \(= 20\).
\\\empty
\(A = \begin{bmatrix}
3  & -2 & 1 & 0 \\
2  & -1 & 3 & 1 \\
2 & -1 & 3 & 4 \\
0 & 1 & 3 & -1 \\
\end{bmatrix}\),
\(X = \begin{bmatrix}
4\\
1\\
-2\\
3
\end{bmatrix}\)
\subsection{{\bfseries\sffamily DONE} \(A_4\)}
\label{sec:org083aa34}
\begin{align*}A_4 &= \begin{vmatrix}
                       3  & -2 & 1 & 4 \\
                       2  & -1 & 3 & 1 \\
                       2 & -1 & 3 & -2 \\
                       0 & 1 & 3 & 3 \\
                     \end{vmatrix}
  \xrightarrow[k_3 = k_3 - k4]{k_4 = k_4 - 3k_2}
  \begin{vmatrix}
    3 & -2 & -3  & 10 \\
    2 & -1 &  2  & 4 \\
    2 & -1 & 5   & 1 \\
    0 & 1  & 0   & 0 \\
  \end{vmatrix}\\
                  &= 1 \cdot (-1)^{6} \cdot \begin{vmatrix}
                                              3 & -3 & 10 \\
                                              2 & 2  & 4  \\
                                              2 & 5  & 1\\
                                              \end{vmatrix}\\
                  &=1 \cdot (6 + 100 - 24) - (40 + 60 -6)\\
                  &=82 - 94\\
                  &= - 12
\end{align*}
\subsection{{\bfseries\sffamily DONE} Podstawianie}
\label{sec:orge43a92f}
$$x_4 = \frac{-12}{20} = \frac{-3}{5}$$
\end{document}