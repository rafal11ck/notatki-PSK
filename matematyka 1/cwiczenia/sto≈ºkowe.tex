% Created 2022-12-04 Sun 17:29
% Intended LaTeX compiler: pdflatex
\documentclass[11pt]{article}
\usepackage[utf8]{inputenc}
\usepackage[T1]{fontenc}
\usepackage{graphicx}
\usepackage{longtable}
\usepackage{wrapfig}
\usepackage{rotating}
\usepackage[normalem]{ulem}
\usepackage{amsmath}
\usepackage{amssymb}
\usepackage{capt-of}
\usepackage{hyperref}
\author{Rafał Grot}
\date{\today}
\title{Stożkowe}
\begin{document}

\maketitle
\tableofcontents

\begin{latex}
\[Q(\vec{x}) = a_{11}x_1^2 + 2a_{12}x_1x_2+a_{22}x_2^}
\to M =
\begin{bmatrix}
    a_{11} & a_{12}\\
    a_{21} & a_{22}\\
\end{bmatrix}\]

\(\det{M}\) -- wyróżnik formy kwadratowej \(Q(\vec{x})\)

\begin{align*}
  \det{M} &> 0 && \text{forma kwadratowa typu eliptycznego}\\
  \det{M} &= 0 && \text{forma kwadratowa typu parabolicznego}\\
  \det{M} &< 0 && \text{forma kwadratowa typu hiperbolicznego}\\
\end{align*}

Sprowadzanie formy kwadratowej do postaci kwadratowej


\[Q(\vec{x}) = a_{11}x_1^2 + 2a_{12}x_1x_2+a_{22}x_2^}
\to
 Q(\vec{x}) = a_{1} \hat{x}_{1}^{2} + a_{2}\hat{x}_2^{2}\]

, gdzie \(a_{1}, a_{2}\) -- wartości własne macierzy \(M\)

\(\hat{x}_1,\hat{x}_{2}\) -- współżędne wektora \(\vec{x}\) w nowej baze ortonormalnej \(\vec{v_{1}}, \vec{v_{2}}\) złożonej z wersorów własnych macierzy \(M\).

wersor własny -- wektor własny o długości 1.
\end{latex}
\end{document}