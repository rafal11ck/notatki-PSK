% Created 2023-03-03 Fri 13:10
% Intended LaTeX compiler: pdflatex
\documentclass[11pt]{article}
\usepackage[utf8]{inputenc}
\usepackage[T1]{fontenc}
\usepackage{graphicx}
\usepackage{longtable}
\usepackage{wrapfig}
\usepackage{rotating}
\usepackage[normalem]{ulem}
\usepackage{amsmath}
\usepackage{amssymb}
\usepackage{capt-of}
\usepackage{hyperref}
\author{placeholder}
\date{\today}
\title{Wyklad01}
\hypersetup{
 pdfauthor={placeholder},
 pdftitle={Wyklad01},
 pdfkeywords={},
 pdfsubject={},
 pdfcreator={Emacs 30.0.50 (Org mode 9.6.1)}, 
 pdflang={English}}
\begin{document}

\maketitle
\tableofcontents

\section{Organizacja}
\label{sec:orgf3a3fa2}
\begin{itemize}
\item mail \texttt{g.slon@tu.kielce.pl}
\item pok 2.27D (\texttt{41 34-24-143})
\item pok 2.07CH (częściej)(\texttt{41 34-24-333})
\item tel komórkowy \texttt{602 350 706}
\end{itemize}
\section{Matma}
\label{sec:org12637b1}
\begin{itemize}
\item egzamin, żęby przysąpić trzeba zaliczyć ćwiczenia.
\item Obesność na wykładzie nie obowiązkowa, będzie lista, nie będzie mieć wpływu na ocenę, chyba żę podciągnąć ocenę.
\item moodle \texttt{Mat2-h4tf} matematyka 2 slon
\end{itemize}
\section{Egzamin}
\label{sec:orga5ca280}
Można używać ściągi, którą wykładowca przyogotuje i tylko takiej.
\begin{itemize}
\item Kalkulator prosty, to złudna nadzieja.
\item Egzamin nie polega na liczeniu.
\item trzeba umieć równanie kwadratowe.
\item umieć tablczkę mnożenia.
\item liczy się zrozumienie problemu.
\end{itemize}
\end{document}